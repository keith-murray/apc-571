\documentclass{article}
\usepackage{amsmath, amssymb}
\usepackage{mathhint}

\begin{document}

% Set the context for this problem - this information appears in the page header
% and helps the hint system understand what material you've covered
\mathhintcontext{
  book=Guckenheimer and Holmes,
  chapter=1,
  section=1.2-1.4,
  problem=2,
  lectures={lecture_3.tex, lecture_4.tex}
}

\begin{problem}
(20 points) \textit{Linear vector fields and maps.}
\begin{enumerate}
  \item[(a)] For the following linear ODE, compute the stable, unstable, and center subspaces $E^s$, $E^u$, $E^c$ of the origin. Sketch these subspaces and some representative trajectories of the system in the phase space. \[ \begin{bmatrix}
    \dot{x}_1 \\ \dot{x}_2 \\ \dot{x}_3
  \end{bmatrix} = \begin{bmatrix}
    1 & -3 & 3 \\ 3 & -5 & 3 \\ 6 & -6 & 4
  \end{bmatrix} \begin{bmatrix}
    x_1 \\ x_2 \\ x_3
  \end{bmatrix} \]
  \item[(b)] For the following linear map on $\mathbb{R}^2$, compute the stable, unstable, and center subspaces $E^s$, $E^u$, $E^c$ of the origin. What are the qualitative differences in the dynamics between $\lambda <0$, $\lambda =0$, and $\lambda >0$? Sketch the subspaces and some representative trajectories of the system in the phase space. \[ \begin{bmatrix}
    x_1 \\ x_2
  \end{bmatrix}\mapsto \begin{bmatrix}
    1 & -1 \\ 0 & \lambda
  \end{bmatrix} \begin{bmatrix}
    x_1 \\ x_2
  \end{bmatrix}, \qquad |\lambda|<1.\]
\end{enumerate}

\end{problem}

\begin{notes}
\paragraph{Part (a)}
Let's start with finding the eigenvalues. It's a slog, but we can write 
\[A-\lambda I= \begin{bmatrix}
    1 & -3 & 3 \\ 3 & -5 & 3 \\ 6 & -6 & 4
  \end{bmatrix} - \lambda I = \begin{bmatrix}
    1-\lambda & -3 & 3 \\ 3 & -5-\lambda & 3 \\ 6 & -6 & 4-\lambda
  \end{bmatrix}  \]
where the eigenvalues are computed as $\operatorname{det}\left(A-\lambda I\right)=0$. Thankfully, we can use the formula for $3\times 3$ matrix determinants of
\[ a_1(b_2c_3-b_3c_2)-a_2(b_1c_3-b_3c_1)+a_3(b_1c_2-b_2c_1) \]
Plugging everything in we get 
\begin{gather*}
  (1-\lambda)\left((-5-\lambda)(4-\lambda)-3(-6)\right)-(-3)\left(3(4-\lambda)-3\cdot 6\right)+(3)\left(3(-6)-(-5-\lambda)\cdot 6\right)\\
  \left((1-\lambda)(-5-\lambda)(4-\lambda)+18(1-\lambda)\right) +\left(9(4-\lambda)-54\right)+\left(-54-18(-5-\lambda)\right)\\
  \left((-5+4\lambda+\lambda^2)(4-\lambda)+18-18\lambda\right) +\left(36-9\lambda-54\right)+\left(-54+90+18\lambda\right)\\
  (-20+21\lambda-\lambda^3)+18-18\lambda +36-9\lambda-54-54+90+18\lambda\\
  \left(-20+18+36-54-54+90\right) + \left(21-18-9+18\right)\lambda +\lambda^3\\
  16+12\lambda+\lambda^3
\end{gather*}
Setting it equal to zero and doing some factorization, we get 
\begin{gather*}
  16+12\lambda+\lambda^3=0\\
  (4-\lambda)(4+4x+x^2)=0\\
  (4-\lambda)(2+\lambda)(2+\lambda)=0
\end{gather*}
where it follows that the eigenvalues are $\lambda_1=4$, $\lambda_2=-2$, and $\lambda_3=-2$.

To find eigenvalues, we can solve for the equation 
\[\begin{bmatrix}
    1-\lambda & -3 & 3 \\ 3 & -5-\lambda & 3 \\ 6 & -6 & 4-\lambda
  \end{bmatrix} \begin{bmatrix}
    x_1 \\ x_2 \\ x_3
  \end{bmatrix} = \begin{bmatrix}
    0 \\ 0 \\ 0
  \end{bmatrix} \]
for each eigenvalue. Starting with $\lambda_1=4$, we have 
\begin{gather*}
  \begin{bmatrix}
    1-4 & -3 & 3 \\ 3 & -5-4 & 3 \\ 6 & -6 & 4-4
  \end{bmatrix} \begin{bmatrix}
    x_1 \\ x_2 \\ x_3
  \end{bmatrix} = \begin{bmatrix}
    0 \\ 0 \\ 0
  \end{bmatrix}\\
  \begin{bmatrix}
    -3 & -3 & 3 \\ 3 & -9 & 3 \\ 6 & -6 & 0
  \end{bmatrix} \begin{bmatrix}
    x_1 \\ x_2 \\ x_3
  \end{bmatrix} = \begin{bmatrix}
    0 \\ 0 \\ 0
  \end{bmatrix}\\
\end{gather*}
where the last row implies that $x_1=x_2$. Plugging that into the expression $-3x_1-3x_2+3x_3=0$, we have 
\begin{gather*}
  -3x_1-3x_1+3x_3=0\\
  -6x_1+3x_3=0\\
  x_3=2x_1
\end{gather*}
Choosing $x_1=1$, we have an eigenvector corresponding to the eigenvalue $\lambda_1=4$ of 
\[ v_1= \begin{bmatrix}
    1 \\ 1 \\ 2
  \end{bmatrix}
\]

Moving on to the eigenvalue $\lambda_{2,3}=-2$, we can write 
\begin{gather*}
  \begin{bmatrix}
    1+2 & -3 & 3 \\ 3 & -5+2 & 3 \\ 6 & -6 & 4+2
  \end{bmatrix} \begin{bmatrix}
    x_1 \\ x_2 \\ x_3
  \end{bmatrix} = \begin{bmatrix}
    0 \\ 0 \\ 0
  \end{bmatrix}\\
  \begin{bmatrix}
    3 & -3 & 3 \\ 3 & -3 & 3 \\ 6 & -6 & 6
  \end{bmatrix} \begin{bmatrix}
    x_1 \\ x_2 \\ x_3
  \end{bmatrix} = \begin{bmatrix}
    0 \\ 0 \\ 0
  \end{bmatrix}\\
\end{gather*}
and since all rows are the same, we are justified in writing
\[x_3=-x_1+x_2\]
where the eigenvectors corresponding to $\lambda_{2,3}=-2$ span this plane. Choosing two arbitrarily, we have 
\[ v_2= \begin{bmatrix}
    0 \\ 1 \\ 1
  \end{bmatrix},\qquad v_3= \begin{bmatrix}
    -1 \\ 0 \\ -1
  \end{bmatrix}
\]

Compiling our work into the stable, unstable, and center subspaces $E^s$, $E^u$, $E^c$, we can write 
\begin{align*}
  E^u&=\operatorname{span}\left\{\begin{bmatrix}
    1 \\ 1 \\ 2
  \end{bmatrix}\right\}\\
  E^s&=\operatorname{span}\left\{\begin{bmatrix}
    0 \\ 1 \\ 1
  \end{bmatrix}, \begin{bmatrix}
    -1 \\ 0 \\ -1
  \end{bmatrix}\right\}\\
  E^c&=\operatorname{span}\left\{0\right\}
\end{align*}

\paragraph{Part (b)}

Okay so we can use SymPy to do the eigenvalues and eigenvectors. Praise be. We have 
\[
\left[ \left( \text{eigenvalue  } 1, \  \text{eigenvector  } \left[\begin{matrix}1\\0\end{matrix}\right]\right), \  \left( \text{eigenvalue  } \lambda, \  \text{eigenvector  } \left[\begin{matrix}- \frac{1}{\lambda - 1}\\1\end{matrix}\right]\right)\right]
\]
Compiling our ``work" into the stable, unstable, and center subspaces $E^s$, $E^u$, $E^c$, we can write 
\begin{align*}
  E^u&=\operatorname{span}\left\{\begin{bmatrix}
  - \frac{1}{\lambda - 1}\\1
  \end{bmatrix}\right\}\\
  E^s&=\operatorname{span}\left\{0\right\}\\
  E^c&=\operatorname{span}\left\{\begin{bmatrix}1\\0\end{bmatrix}\right\}
\end{align*}

\mathhint{Nudge}{2026-02-10 22:41}{You're in great shape on part (a). For part (b), remember that the stability criteria for maps differ from flows — think carefully about what determines whether an eigenvalue corresponds to $E^s$, $E^u$, or $E^c$ for a *discrete* map, and how the sign/magnitude of $\lambda$ affects which subspace each eigenvector belongs to.}

Ah yes that's right. I assigned the eigenvector to the unstable subspace instead of stable because $|\lambda|<1$
\begin{align*}
  E^s&=\operatorname{span}\left\{\begin{bmatrix}
  - \frac{1}{\lambda - 1}\\1
  \end{bmatrix}\right\}\\
  E^u&=\operatorname{span}\left\{0\right\}\\
  E^c&=\operatorname{span}\left\{\begin{bmatrix}1\\0\end{bmatrix}\right\}
\end{align*}


\end{notes}

\begin{solution}
% Write your solution here
% Present your final, clean solution.

\end{solution}


\end{document}
