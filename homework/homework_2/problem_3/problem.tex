\documentclass{article}
\usepackage{amsmath, amssymb}
\usepackage{tikz-cd}
\usepackage{mathhint}

\begin{document}

% Set the context for this problem - this information appears in the page header
% and helps the hint system understand what material you've covered
\mathhintcontext{
  book=Guckenheimer and Holmes,
  chapter=1,
  section=1.2-1.4,
  problem=2,
  lectures={lecture_3.tex, lecture_4.tex}
}

\begin{problem}
(38 points) For the following dynamical systems, find the fixed point(s) and compute the associated stable and unstable manifolds, up to fourth order (that is, if you consider an approximation $y=h(x)$, let $h$ be a quartic polynomial). Use these results to sketch the global phase portraits.
\begin{enumerate}
  \item[(a)] For the following ODE, assume $\lambda>0$.\begin{align*}
    \dot{x}&=x(1-x^2)\\
    \dot{y}&=-\lambda(y-x^2).
  \end{align*}
  \item[(b)] For the following map, assume $|\lambda|<1$ and $|\mu|>1$.\begin{align*}
    x&\mapsto \lambda x,\\
    y&\mapsto \mu y-x^3.
  \end{align*}
\end{enumerate}

\end{problem}

\begin{notes}
\paragraph{Part (A)}
To begin, let's find the fixed points by writing 
\begin{align*}
  0&=\dot{x}=x(1-x^2)\implies x=0,x=\pm1\\
  0&=\dot{y}=-\lambda (y-x^2) \implies y=x^2
\end{align*}
Hence our fixed points are $(0,0),(-1,1),(1,1)$. To linearize, we can compute the Jacobian 
\[ J_f(x,y) = \begin{bmatrix}
\frac{\partial f_1}{\partial x} & \frac{\partial f_1}{\partial y} \\
\frac{\partial f_2}{\partial x} & \frac{\partial f_2}{\partial y} 
\end{bmatrix} \]
with the quatities being 
\begin{gather*}
  \frac{\partial f_1}{\partial x}=1-3x^2\\
  \frac{\partial f_1}{\partial y}=0\\
  \frac{\partial f_2}{\partial x}=2\lambda x\\
  \frac{\partial f_2}{\partial y}=-\lambda
\end{gather*}
Now we can compute our linearized matrices 
\begin{gather*}
  Df(0,0)=\begin{bmatrix}
    1 & 0\\
    0 & -\lambda
  \end{bmatrix}\\
  Df(-1,1)=\begin{bmatrix}
    -2 & 0\\
    -2\lambda & -\lambda
  \end{bmatrix}\\
  Df(1,1)=\begin{bmatrix}
    -2 & 0\\
    2\lambda & -\lambda
  \end{bmatrix}\\
\end{gather*}

Let's first approximate the stable and unstable manifolds associated with the $(0,0)$ fixed point. Our general workflow will be to calculate the eigenvalues and vectors, construct $h(x)$ or $g(y)$, then differentiate, group coefficients together, and solve for the coefficients.

\mathhint{Nudge}{2026-02-11 08:41}{You're off to a great start — the fixed points, Jacobian, and linearizations all look correct. Keep going with your plan for the origin: identify the stable and unstable eigenspaces and then look for the invariant manifolds as graphs over those subspaces.}

For our fixed point $(0,0)$, we have 
\[
\left[ \left(\text{eigenvalue  }  1, \ \text{eigenvector  }   \left[\begin{matrix}1\\0\end{matrix}\right]\right), \  \left(\text{eigenvalue  }  - \lambda, \  \text{eigenvector  }  \left[\begin{matrix}0\\1\end{matrix}\right]\right)\right]
\]
Starting with our unstable manifold with the eigenvalue of $1$ and eigenvector of $\left[\begin{matrix}1\\0\end{matrix}\right]$, we can write 
\[ y=h(x)=a_0 +a_1x+a_2x^2+a_3x^3+a_4x^4 \]
From our eigenvector we can deduce that $a_0=a_1=0$. Taking the time derivative, we have 
\begin{gather*}
  \dot{y}=h'(x)=\left(2a_2x+3a_3x^2+4a_4x^3\right)\dot{x}\\
  -\lambda(y-x^2)=\left(2a_2x+3a_3x^2+4a_4x^3\right)x(1-x^2)\\
  -\lambda(a_2x^2+a_3x^3+a_4x^4-x^2)=\left(2a_2x+3a_3x^2+4a_4x^3\right)x(1-x^2)\\
\end{gather*}
Is this right? This feels like a heinous expression.

\mathhint{Nudge}{2026-02-11 09:07}{You're on exactly the right track — that expression is correct and not as bad as it looks. Just expand both sides, collect terms by powers of $x$, and match coefficients order by order; you'll find each $a_k$ is determined cleanly one at a time.}

Fair enough. I'm gonnna save some time though and just parse with SymPy. Hence, we get 
\[
 \left(- a_{2} \lambda - 2 a_{2} + \lambda\right)x^{2} + \left(- a_{3} \lambda - 3 a_{3}\right) x^{3} + \left(2 a_{2} - a_{4} \lambda - 4 a_{4}\right)  x^{4} + 3 a_{3} x^{5} + 4 a_{4} x^{6} =0
\]
where we can ignore terms $3 a_{3} x^{5} + 4 a_{4} x^{6}$ since we are interested in 4th order approximations. Is this right?


\mathhint{Nudge}{2026-02-11 11:29}{You're almost there — just set the coefficient of each power of $x$ to zero and solve sequentially for $a_2$, $a_3$, $a_4$. It should come out very cleanly!}

Do we get
\begin{align*}
  a_2&=\frac{\lambda}{\lambda+2}\\
  a_3&=0 \qquad\text{since $\lambda>0$}\\
  a_4&=\frac{2a_2}{\lambda+4}
\end{align*}

\mathhint{Nudge}{2026-02-11 11:38}{You're on the right track — those coefficients are correct! Now simplify $a_4$ by substituting your expression for $a_2$, then move on to the stable manifold at the origin and the other fixed points.}

Simplifying $a_4$, we get 
\begin{gather*}
  a_4=\frac{\lambda}{\lambda+2}\cdot\frac{2}{\lambda+4}\\
  a_4 = \frac{2\lambda}{(\lambda+2)(\lambda+4)}
\end{gather*}
and hence our approximated unstable manifold is 
\[ h(x)= \frac{\lambda}{\lambda+2}x^2 + \frac{2\lambda}{(\lambda+2)(\lambda+4)}x^4 \]

Moving on to the stable manifold with the eigenvalue of $-\lambda$ and eigenvector of $\left[\begin{matrix}0\\1\end{matrix}\right]$, note that for any point where $x=0$, we have the following system 
\begin{align*}
  \dot{x}&=0\\
  \dot{y}&=-\lambda y
\end{align*}
implying that trajctories that start with $x=0$ have $\lim_{t\to\infty}y(t)\to 0$, inline with the interpretation of a stable manifold. Hence, the stable manifold is simply an extension of the eigenvector and can be described by the line 
\[ x=0 \] 
This reasoning feels right, but is it in the spirit of the problem?

\mathhint{Nudge}{2026-02-11 13:28}{Your reasoning is perfectly correct and well-justified — the $y$-axis is indeed the exact stable manifold, so there's no need for a polynomial approximation. That's absolutely in the spirit of the problem. Now tackle the manifolds at $(\pm 1, 1)$.}

Let's move on to the manifolds for $(-1,1)$. The eigenvalues and eigenvectors are 
\[
\left[ \left(\text{eigenvalue  } -2, \ \text{eigenvector  }  \left[\begin{matrix}\frac{2 - \lambda}{2 \lambda}\\1\end{matrix}\right]\right), \  \left(\text{eigenvalue  } - \lambda, \ \text{eigenvector  } \left[\begin{matrix}0\\1\end{matrix}\right]\right)\right]
\]

For the manifold corresponding to the eigenvalue $-2$, we can write 
\[ y=h(x)=a_0 +a_1x+a_2x^2+a_3x^3+a_4x^4 \]
where I think $a_1=\frac{2\lambda}{2-\lambda}$, but that doesn't feel right.

\mathhint{Nudge}{2026-02-11 15:24}{You're on the right track thinking about the eigenvector slope, but remember you need to work in shifted coordinates centered at $(-1, 1)$ — let $u = x + 1$ and $v = y - 1$, then express the system in $(u, v)$ and apply the same manifold computation technique there.}

I guess it's a taylor series approximation, so really it is 
\[ y=h(x)=1 +a_1(x+1)+a_2(x+1)^2+a_3(x+1)^3+a_4(x+1)^4 \]
Now let's take the derivative to get 
\[ \dot{y}=h'(x)=\left(a_1+2a_2(x+1)+3a_3(x+1)^2+4a_4(x+1)^3\right)\dot{x} \]
and substituting in for $\dot{x},\dot{y}$, we get 
\[ -\lambda(y-x^2) = \left(a_1+2a_2(x+1)+3a_3(x+1)^2+4a_4(x+1)^3\right)x(1-x^2) \]
Substituting in $h(x)$ for $y$, we have 
\begin{align*}
  -\lambda(1 +a_1(x+1)+a_2(x+1)^2&+a_3(x+1)^3+a_4(x+1)^4-x^2)\\
  &= \left(a_1+2a_2(x+1)+3a_3(x+1)^2+4a_4(x+1)^3\right)x(1-x^2)
\end{align*}

\mathhint{Nudge}{2026-02-11 18:55}{You're heading in the right direction, but working in the original $x$ variable will get messy fast. It'll be much cleaner if you actually substitute $u = x+1$ (so $x = u-1$) and rewrite everything in terms of $u$ before expanding — that way you're matching powers of $u$ near the origin, just like you did for the first fixed point.}

To make matters easier, we will substitute in $u=x+1$ (also $x=u-1$) to get 
\begin{align*}
  -\lambda(1 +a_1(u)+a_2(u)^2&+a_3(u)^3+a_4(u)^4-(u-1)^2)\\
  &= \left(a_1+2a_2(u)+3a_3(u)^2+4a_4(u)^3\right)(u-1)(1-(u-1)^2)
\end{align*}
Welp, now let's turn to SymPy. 
\begin{align*}
  4 a_{4} u^{6} &+ u^{5} \left(3 a_{3} - 12 a_{4}\right) + u^{4} \left(2 a_{2} - 9 a_{3} - a_{4} \lambda + 8 a_{4}\right) \\
  &+ u^{3} \left(a_{1} - 6 a_{2} - a_{3} \lambda + 6 a_{3}\right) + u^{2} \left(- 3 a_{1} - a_{2} \lambda + 4 a_{2} + \lambda\right) \\
  &+ u \left(- a_{1} \lambda + 2 a_{1} - 2 \lambda\right)=0
\end{align*}
Gross. Now it's time to drop the higher order terms, and solve for the coefficients 
\begin{align*}
  a_1&=\frac{2\lambda}{2-\lambda}\\
  a_2&=\frac{3a_1-\lambda}{4-\lambda}\\
  a_3&=\frac{6a_2-a_1}{6-\lambda}\\
  a_4&=\frac{9a_3-2a_2}{8-\lambda}
\end{align*}
I will push these through later, but this looks right?

\mathhint{Nudge}{2026-02-11 19:15}{You're doing great — those recurrence relations look correct. Now just substitute forward to get explicit expressions in terms of $\lambda$, and then move on to the other eigenvalue's manifold at $(-1,1)$ and the fixed point $(1,1)$.}

For the manifold corresponding to the eigenvalue $-\lambda$, we can use a similar argument as before to reason that the stable manifold is $x=-1$. Namely, if $x=-1$, then our system becomes 
\begin{align*}
  \dot{x}&=-1\cdot(1-(-1)^2)=0\\
  \dot{y}&=-\lambda(y-(-1)^2)=-\lambda y-\lambda
\end{align*}
which will just flow to $(-1,1)$.

\mathhint{Nudge}{2026-02-11 19:40}{You're on the right track with that argument, but double-check your algebra — substituting $x = -1$ gives $\dot{y} = -\lambda(y - 1)$, not $-\lambda y - \lambda$.}

You are right! Namely, if $x=-1$, then our system becomes 
\begin{align*}
  \dot{x}&=-1\cdot(1-(-1)^2)=0\\
  \dot{y}&=-\lambda(y-(-1)^2)=-\lambda y+\lambda
\end{align*}
which will just flow to $(-1,1)$ since $\lambda>0$.

\mathhint{Nudge}{2026-02-11 19:41}{You've corrected the algebra issue — $\dot{y} = -\lambda(y-1)$ indeed flows to $y=1$, confirming $x=-1$ is the stable manifold at $(-1,1)$. Now use symmetry to handle $(1,1)$ efficiently, and then move on to part (b).}

Ah yes, intuitively we would just mirror our previous manifolds at $(-1,1)$ to get the manifolds at $(1,1)$. To use symmetry, don't we have to argue that our system is symmetric? Wait, no. I guess I just have to argue that the particular manifold would be symmetric.

Let's first compute our eigenvalues and eigenvectors at $(1,1)$
\[
\left[ \left( \text{eigenvalue  }-2, \  \text{eigenvector  }\left[\begin{matrix}\frac{\lambda - 2}{2 \lambda}\\1\end{matrix}\right]\right), \  \left(\text{eigenvalue  } - \lambda, \ \text{eigenvector  }\left[\begin{matrix}0\\1\end{matrix}\right]\right)\right]
\]
Notice that the eigenvector $\left[\begin{matrix}\frac{\lambda - 2}{2 \lambda}\\1\end{matrix}\right]$ corresponding to the eigenvalue $-2$ is symmetric about the y-axis to the eigenvector corresponding to the eigenvalue $-2$ for the fixed point $(-1,1)$. Hence, given that our fixed point is also symmetric about the y-axis, it follows that the manifolds are also symmetric about the y-axis. Therefore, our manifold corresponding to eigenvalue $-2$ is 
\[ y=h(-x)=1 +a_1(-x+1)+a_2(-x+1)^2+a_3(-x+1)^3+a_4(-x+1)^4 \]
with coefficients
\begin{align*}
  a_1&=\frac{2\lambda}{2-\lambda}\\
  a_2&=\frac{3a_1-\lambda}{4-\lambda}\\
  a_3&=\frac{6a_2-a_1}{6-\lambda}\\
  a_4&=\frac{9a_3-2a_2}{8-\lambda}
\end{align*}
and our manifold corresponing to eigenvalue $-\lambda$ is $x=+1$.

\mathhint{Nudge}{2026-02-11 20:04}{You're making great progress — now use the symmetry of the system (note how $\dot{x}$ is an odd function of $x$ and $\dot{y}$ depends on $x^2$) to quickly write down the manifolds at $(1,1)$ without redoing the full computation, and then move on to part (b).}

Ah yes, in general, a system is y-axis symmetric if for $\dot{x}=f(x,y)$ and $\dot{y}=g(x,y)$, then $f(-x,y)=-f(x,y)$ and $g(-x,y)=g(x,y)$. For our system, we can write 
\[f(-x,y)=-x(1-(-x)^2)=-x(1-x^2)=-f(x,y)\]
and 
\[g(-x,y)=-\lambda(y-(-x)^2)=-\lambda(y-x^2)=g(x,y)\]
Hence our system is symmetric about the y-axis, therefore our manifold approximations must be as well. Therefore, our manifold corresponding to eigenvalue $-2$ is 
\[ y=h(-x)=1 +a_1(-x+1)+a_2(-x+1)^2+a_3(-x+1)^3+a_4(-x+1)^4 \]
with coefficients
\begin{align*}
  a_1&=\frac{2\lambda}{2-\lambda}\\
  a_2&=\frac{3a_1-\lambda}{4-\lambda}\\
  a_3&=\frac{6a_2-a_1}{6-\lambda}\\
  a_4&=\frac{9a_3-2a_2}{8-\lambda}
\end{align*}
and our manifold corresponing to eigenvalue $-\lambda$ is $x=+1$.


\mathhint{Nudge}{2026-02-11 20:32}{You've completed part (a) nicely — now it's time to move on to part (b) with the map, where the invariant manifold condition will look a bit different since you're working with a discrete map rather than a flow.}

\paragraph{Part (B)}

To begin, let's find the fixed points by writing 
\begin{align*}
  x&=\lambda x\\
\end{align*}
where since $|\lambda|<1$, it follows that the only fixed $x$ coord is $x=0$. For $y$, we can write 
\[ y=\mu y \]
where since $|\mu|>1$, it follows that the only fixed $y$ coord is $y=0$. Hence the only fixed point is $(0,0)$.

\mathhint{Nudge}{2026-02-11 20:54}{You're on the right track — now identify the stable and unstable directions at the origin and set up the invariant manifold condition using the map (i.e., $h(\lambda x) = \mu\, h(x) - x^3$ or similar) instead of the flow condition you used in part (a).}
To linearize, we can compute the Jacobian 
\[ J_f(x,y) = \begin{bmatrix}
\frac{\partial f_1}{\partial x} & \frac{\partial f_1}{\partial y} \\
\frac{\partial f_2}{\partial x} & \frac{\partial f_2}{\partial y} 
\end{bmatrix} \]
with the quatities being 
\begin{gather*}
  \frac{\partial f_1}{\partial x}=\lambda\\
  \frac{\partial f_1}{\partial y}=0\\
  \frac{\partial f_2}{\partial x}=-3x^2\\
  \frac{\partial f_2}{\partial y}=\mu
\end{gather*}
Now we can compute our linearized matrix 
\[
Df(0,0)=\begin{bmatrix}
  \lambda & 0\\
  0 & \mu
\end{bmatrix}\\
\]
where are eigenvalues and eigenvectors are 
\[
\left[ \left( \text{eigenvalue  }\lambda, \  \text{eigenvector  }\left[\begin{matrix}1\\0\end{matrix}\right]\right), \  \left(\text{eigenvalue  } \mu, \ \text{eigenvector  }\left[\begin{matrix}0\\1\end{matrix}\right]\right)\right]
\]

\mathhint{Nudge}{2026-02-11 21:14}{You're on the right track with the eigenvalues and eigenvectors. Now think about which direction is stable and which is unstable (given the assumptions on $|\lambda|$ and $|\mu|$), and set up the invariant manifold as a graph $y = h(x)$ over the appropriate eigenspace.}

Given $|\lambda|<1$, we can construct our stable manifold by writing 
\[ y=h(x)=a_0 +a_1x+a_2x^2+a_3x^3+a_4x^4 \]
where $a_0=0$ since we are at the origin and $a_1=0$ since the corresponding eigenvector is $\left[\begin{matrix}1\\0\end{matrix}\right]$. Hence we have 
\[ y=h(x)=a_2x^2+a_3x^3+a_4x^4 \]
My guess is that we have something like 
\begin{gather*}
  y=h(x)=a_2x^2+a_3x^3+a_4x^4\\
  y-x^3=a_2x^2+a_3x^3+a_4x^4\\
  h(x)-x^3=a_2x^2+a_3x^3+a_4x^4\\
\end{gather*}
but this feels circular.

\mathhint{Nudge}{2026-02-11 21:22}{You're close — think about what invariance means for a *map*: if a point $(x, h(x))$ is on the manifold, then its *image* under the map must also be on the manifold, giving you $h(\lambda x) = \mu\, h(x) - x^3$.}

Oh this makes sense! It's kind of like a commute diagram. More specifically 
\[
\begin{tikzcd}
x_t \arrow{r}{h} \arrow{d}{f} & y_t \arrow{d}{g} \\
x_{t+1} \arrow{r}{h} & y_{t+1}
\end{tikzcd}
\]
where $x\mapsto f(x,y)$ and $y\mapsto g(x,y)$. Hence,
\[ h(f(x,y))=g(x,h(x)) \]
and plugging in our expressions, we have 
\begin{gather*}
  h(\lambda x) = \mu\, h(x) - x^3\\
  a_2\lambda^2x^2+a_3\lambda^3x^3+a_4\lambda^4=\mu\left(a_2x^2+a_3x^3+a_4x^4\right)-x^3
\end{gather*}
where grouping terms together gives us 
\begin{align*}
  a_2&=0\\
  a_3&=\frac{1}{\mu-\lambda^3}\\
  a_4&=0
\end{align*}
because $a\lambda^n=a\mu$ implies $a=0$.

\mathhint{Nudge}{2026-02-11 23:18}{You've got the stable manifold for the map — nice work! Now set up the unstable manifold as a graph $x = g(y)$ over the unstable eigenspace and apply the same invariance logic.}

Turning our attention to approximating the unstable manifold $x=h^u(y)$, then we can write 
\[x=a_2y^2+a_3y^3+a_4y^4\]
where $a_0=0$ since we are at the origin and $a_1=0$ since the corresponding eigenvector is $\left[\begin{matrix}0\\1\end{matrix}\right]$. Hence we need to satisfy the following equation
\[ \lambda h^u(y)=h^u(\mu y - h^u(y)^3) \]
which seems incredibly involved.

Could we not just argue that the unstable manifold is $x=0$ since our map when $x=0$ becomes 
\begin{align*}
    x&\mapsto 0,\\
    y&\mapsto \mu y - 0 = \mu y.
\end{align*}
and we merely just stay along the y-axis? This is the same argument that we used in part (a).

\mathhint{Nudge}{2026-02-12 07:49}{You're exactly right — the $y$-axis ($x=0$) is invariant under the map and points on it grow as $|\mu|>1$, so it is indeed the exact unstable manifold. Nice use of the same reasoning from part (a)!}


\end{notes}

\begin{solution}
\paragraph{Part (A)}
To begin, let's find the fixed points by writing 
\begin{align*}
  0&=\dot{x}=x(1-x^2)\implies x=0,x=\pm1\\
  0&=\dot{y}=-\lambda (y-x^2) \implies y=x^2
\end{align*}
Hence our fixed points are $(0,0),(-1,1),(1,1)$. To linearize, we can compute the Jacobian 
\[ J_f(x,y) = \begin{bmatrix}
\frac{\partial f_1}{\partial x} & \frac{\partial f_1}{\partial y} \\
\frac{\partial f_2}{\partial x} & \frac{\partial f_2}{\partial y} 
\end{bmatrix} \]
with the quatities being 
\begin{gather*}
  \frac{\partial f_1}{\partial x}=1-3x^2\\
  \frac{\partial f_1}{\partial y}=0\\
  \frac{\partial f_2}{\partial x}=2\lambda x\\
  \frac{\partial f_2}{\partial y}=-\lambda
\end{gather*}
Now we can compute our linearized matrices 
\begin{gather*}
  Df(0,0)=\begin{bmatrix}
    1 & 0\\
    0 & -\lambda
  \end{bmatrix}\\
  Df(-1,1)=\begin{bmatrix}
    -2 & 0\\
    -2\lambda & -\lambda
  \end{bmatrix}\\
  Df(1,1)=\begin{bmatrix}
    -2 & 0\\
    2\lambda & -\lambda
  \end{bmatrix}\\
\end{gather*}

Let's first approximate the stable and unstable manifolds associated with the $(0,0)$ fixed point. Our general workflow will be to calculate the eigenvalues and vectors, construct $h(x)$ or $g(y)$, then differentiate, group coefficients together, and solve for the coefficients.

For our fixed point $(0,0)$, we have 
\[
\left[ \left(\text{eigenvalue  }  1, \ \text{eigenvector  }   \left[\begin{matrix}1\\0\end{matrix}\right]\right), \  \left(\text{eigenvalue  }  - \lambda, \  \text{eigenvector  }  \left[\begin{matrix}0\\1\end{matrix}\right]\right)\right]
\]

Starting with our unstable manifold with the eigenvalue of $1$ and eigenvector of $\left[\begin{matrix}1\\0\end{matrix}\right]$, we can write 
\[ y=h(x)=a_0 +a_1x+a_2x^2+a_3x^3+a_4x^4 \]
From our eigenvector we can deduce that $a_0=a_1=0$. Taking the time derivative, we have 
\begin{gather*}
  \dot{y}=h'(x)=\left(2a_2x+3a_3x^2+4a_4x^3\right)\dot{x}\\
  -\lambda(y-x^2)=\left(2a_2x+3a_3x^2+4a_4x^3\right)x(1-x^2)\\
  -\lambda(a_2x^2+a_3x^3+a_4x^4-x^2)=\left(2a_2x+3a_3x^2+4a_4x^3\right)x(1-x^2)\\
\end{gather*}

I'm gonnna save some time and just parse with SymPy. Hence, we get 
\[
 \left(- a_{2} \lambda - 2 a_{2} + \lambda\right)x^{2} + \left(- a_{3} \lambda - 3 a_{3}\right) x^{3} + \left(2 a_{2} - a_{4} \lambda - 4 a_{4}\right)  x^{4} + 3 a_{3} x^{5} + 4 a_{4} x^{6} =0
\]
where we can ignore terms $3 a_{3} x^{5} + 4 a_{4} x^{6}$ since we are interested in 4th order approximations. Now we get
\begin{align*}
  a_2&=\frac{\lambda}{\lambda+2}\\
  a_3&=0 \qquad\text{since $\lambda>0$}\\
  a_4&=\frac{2a_2}{\lambda+4}
\end{align*}
Simplifying $a_4$, we get 
\begin{gather*}
  a_4=\frac{\lambda}{\lambda+2}\cdot\frac{2}{\lambda+4}\\
  a_4 = \frac{2\lambda}{(\lambda+2)(\lambda+4)}
\end{gather*}
and hence our approximated unstable manifold is 
\[ h(x)= \frac{\lambda}{\lambda+2}x^2 + \frac{2\lambda}{(\lambda+2)(\lambda+4)}x^4 \]


Moving on to the stable manifold with the eigenvalue of $-\lambda$ and eigenvector of $\left[\begin{matrix}0\\1\end{matrix}\right]$, note that for any point where $x=0$, we have the following system 
\begin{align*}
  \dot{x}&=0\\
  \dot{y}&=-\lambda y
\end{align*}
implying that trajctories that start with $x=0$ have $\lim_{t\to\infty}y(t)\to 0$, inline with the interpretation of a stable manifold. Hence, the stable manifold is simply an extension of the eigenvector and can be described by the line 
\[ x=0 \] 

Let's move on to the manifolds for $(-1,1)$. The eigenvalues and eigenvectors are 
\[
\left[ \left(\text{eigenvalue  } -2, \ \text{eigenvector  }  \left[\begin{matrix}\frac{2 - \lambda}{2 \lambda}\\1\end{matrix}\right]\right), \  \left(\text{eigenvalue  } - \lambda, \ \text{eigenvector  } \left[\begin{matrix}0\\1\end{matrix}\right]\right)\right]
\]
This is nice because both of our eigenvalues are negative and their eigenvectors are linearly independent. Hence our $E^s$ for $(-1,1)$ is all of $\mathbb{R}^2$, and the stable manifold theorem implies that our manifold approximation will be tangent to $\mathbb{R}^2$ near $(-1,1)$. Given that our system is in $\mathbb{R}^2$, an approximation of the stable manifold would be any neighborhood $U\subset \mathbb{R}^2$ as long as $U$ does not intersect $x=0$, the stable manifold of $(0,0)$. Hence, the region
\[U=\left\{ (x,y)\in\mathbb{R}^2: y\in\mathbb{R},\ x\in(-\infty,0) \right\}\]
is the stable manifold of the fixed point $(-1,1)$. There is no unstable manifold.

Moving on to the manifolds for $(1,1)$, the eigenvalues and eigenvectors are 
\[
\left[ \left( \text{eigenvalue  }-2, \  \text{eigenvector  }\left[\begin{matrix}\frac{\lambda - 2}{2 \lambda}\\1\end{matrix}\right]\right), \  \left(\text{eigenvalue  } - \lambda, \ \text{eigenvector  }\left[\begin{matrix}0\\1\end{matrix}\right]\right)\right]
\]
This is also nice because it mirrors are situation for the fixed point $(-1,1)$. Hence, our same argument applies, and the manifold is the region 
\[U=\left\{ (x,y)\in\mathbb{R}^2: y\in\mathbb{R},\ x\in(0,\infty) \right\}\]
where the valid region of our $x$ coordinates are fliped to accomodate our fixed point $(1,1)$.

Here is the sketch of our global phase portrait:

\includegraphics[width=0.5\textwidth]{sketch/problem_3_a.png}

Note that we can exactly solve for the trajectories at $x=-1$ and $x=1$ since they imply 
\begin{align*}
  \dot{x}&=0\\
  \dot{y}&=-\lambda (y-1)
\end{align*}
where $y$ will simply flow to $y=1$.

\paragraph{Part (B)}

To begin, let's find the fixed points by writing 
\begin{align*}
  x&=\lambda x\\
\end{align*}
where since $|\lambda|<1$, it follows that the only fixed $x$ coord is $x=0$. For $y$, we can write 
\[ y=\mu y \]
where since $|\mu|>1$, it follows that the only fixed $y$ coord is $y=0$. Hence the only fixed point is $(0,0)$.

To linearize, we can compute the Jacobian 
\[ J_f(x,y) = \begin{bmatrix}
\frac{\partial f_1}{\partial x} & \frac{\partial f_1}{\partial y} \\
\frac{\partial f_2}{\partial x} & \frac{\partial f_2}{\partial y} 
\end{bmatrix} \]
with the quatities being 
\begin{gather*}
  \frac{\partial f_1}{\partial x}=\lambda\\
  \frac{\partial f_1}{\partial y}=0\\
  \frac{\partial f_2}{\partial x}=-3x^2\\
  \frac{\partial f_2}{\partial y}=\mu
\end{gather*}
Now we can compute our linearized matrix 
\[
Df(0,0)=\begin{bmatrix}
  \lambda & 0\\
  0 & \mu
\end{bmatrix}\\
\]
where are eigenvalues and eigenvectors are 
\[
\left[ \left( \text{eigenvalue  }\lambda, \  \text{eigenvector  }\left[\begin{matrix}1\\0\end{matrix}\right]\right), \  \left(\text{eigenvalue  } \mu, \ \text{eigenvector  }\left[\begin{matrix}0\\1\end{matrix}\right]\right)\right]
\]

Given $|\lambda|<1$, we can construct our stable manifold by writing 
\[ y=h(x)=a_0 +a_1x+a_2x^2+a_3x^3+a_4x^4 \]
where $a_0=0$ since we are at the origin and $a_1=0$ since the corresponding eigenvector is $\left[\begin{matrix}1\\0\end{matrix}\right]$. Hence we have 
\[ y=h(x)=a_2x^2+a_3x^3+a_4x^4 \]
To compute the other coefficients, we can use the following commute diagram that is implied given our map is invariant under our map:
\[
\begin{tikzcd}
x_t \arrow{r}{h} \arrow{d}{f} & y_t \arrow{d}{g} \\
x_{t+1} \arrow{r}{h} & y_{t+1}
\end{tikzcd}
\]
where $x\mapsto f(x,y)$ and $y\mapsto g(x,y)$. Hence,
\[ h(f(x,y))=g(x,h(x)) \]
and plugging in our expressions, we have 
\begin{gather*}
  h(\lambda x) = \mu\, h(x) - x^3\\
  a_2\lambda^2x^2+a_3\lambda^3x^3+a_4\lambda^4=\mu\left(a_2x^2+a_3x^3+a_4x^4\right)-x^3
\end{gather*}
where grouping terms together gives us 
\begin{align*}
  a_2&=0\\
  a_3&=\frac{1}{\mu-\lambda^3}\\
  a_4&=0
\end{align*}
because $a\lambda^n=a\mu$ implies $a=0$. Hence our stable manifold approximation is 
\[ h(x)=\frac{1}{\mu-\lambda^3}x^3 \]

Turning our attention to approximating the unstable manifold $x=h^u(y)$, we can argue that the unstable manifold is $x=0$ since our map when $x=0$ becomes 
\begin{align*}
    x&\mapsto 0,\\
    y&\mapsto \mu y - 0 = \mu y.
\end{align*}
and we merely just stay along the y-axis.

Below are the various global phase portrait sketches with example trajectories. Note that there are four conditions, one for each pairing of positive and negative $\mu$ and $\lambda$ values.

\includegraphics[width=0.45\textwidth]{sketch/p3b_gg.png}
\includegraphics[width=0.45\textwidth]{sketch/p3b_lg.png}

\includegraphics[width=0.45\textwidth]{sketch/p3b_gl.png}
\includegraphics[width=0.45\textwidth]{sketch/p3b_ll.png}


\end{solution}

\end{document}
