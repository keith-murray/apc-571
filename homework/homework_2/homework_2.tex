\documentclass{article}
\usepackage{amsmath, amssymb}
\usepackage{pgfplots}
\usepackage{tikz-cd}
\usepackage{mathwriteup}

\begin{document}

% Set the context for this problem - this information appears in the page header
% and helps the hint system understand what material you've covered
\mathwriteupcontext{
  author=Keith Murray,
  assignment=Homework 2,
  disclosure=Generative AI was used\\ Collaborated with Jeremy S. and Sasha L.
}

\begin{problem}
(42 points) (Modification of Ex. 1.3.2 in GH) For each of the following systems, find all of the fixed points and classify their local stability using linearization. At each fixed point, compute eigenvalues and eigenvectors of the linearization and sketch (by hand) the trajectories of the local flow near the fixed points. Then use a computer to plot sample trajectories and sketch the global phase portrait. [\textit{Hint:} Be sure to first rewrite second-order equations as first-order systems as we did in lecture, and in each case consider $\epsilon>0$, $\epsilon=0$, and $\epsilon<0$.]
\begin{enumerate}
  \item[(a)] $\ddot{x}+\epsilon\dot{x}-x+x^3=0$
  \item[(b)] $\ddot{x}+\epsilon\dot{x}^3+\sin x=0$
  \item[(c)] $\dot{x}=-x+x^2$, $\dot{y}=x+y$
\end{enumerate}

\end{problem}


\begin{solution}
\paragraph{Part (A)}
Here is the rewrite of our system 
\begin{align*}
  \dot{x}&=y\\
  \dot{y}&=x-x^3-\epsilon y
\end{align*}
To find the fixed points, we can set our system equal to zero getting 
\begin{align*}
  0&=\dot{x}=y\\
  0&=\dot{y}=x-x^3-\epsilon y=x-x^3=x(1-x^2)
\end{align*}
where it becomes clear that our fixed points are $(0,0),(-1,0),(1,0)$. To linearize, we can compute the Jacobian 
\[ J_f(x,y) = \begin{bmatrix}
\frac{\partial f_1}{\partial x} & \frac{\partial f_1}{\partial y} \\
\frac{\partial f_2}{\partial x} & \frac{\partial f_2}{\partial y} 
\end{bmatrix} \]
with the quatities being 
\begin{gather*}
  \frac{\partial f_1}{\partial x}=0\\
  \frac{\partial f_1}{\partial y}=1\\
  \frac{\partial f_2}{\partial x}=1-3x^2\\
  \frac{\partial f_2}{\partial y}=-\epsilon
\end{gather*}
Now we can compute our linearized matrices 
\begin{gather*}
  Df(0,0)=\begin{bmatrix}
    0 & 1\\
    1 & -\epsilon
  \end{bmatrix}\\
  Df(-1,0)=\begin{bmatrix}
    0 & 1\\
    -2 & -\epsilon
  \end{bmatrix}\\
  Df(1,0)=\begin{bmatrix}
    0 & 1\\
    -2 & -\epsilon
  \end{bmatrix}\\
\end{gather*}

Via SymPy (which we are allowed to use), we have the following eigenvalues and vectors for fixed point $(0,0)$
\[\left[ \left(\lambda^-_0= - \frac{\epsilon}{2} - \frac{\sqrt{\epsilon^{2} + 4}}{2},  \  v^-_0=\left[\begin{matrix}\frac{\epsilon}{2} - \frac{\sqrt{\epsilon^{2} + 4}}{2}\\1\end{matrix}\right]\right), \  \left(\lambda^+_0= - \frac{\epsilon}{2} + \frac{\sqrt{\epsilon^{2} + 4}}{2},  \  v^+_0=\left[ \begin{matrix}\frac{\epsilon}{2} + \frac{\sqrt{\epsilon^{2} + 4}}{2}\\1\end{matrix}\right]\right)\right]
\]
and for the $(-1,0),(1,0)$ we have 
\[
\left[ \left( \lambda^-_{\pm1}=- \frac{\epsilon}{2} - \frac{\sqrt{\epsilon^{2} - 8}}{2}, \ v^-_{\pm1} = \left[\begin{matrix}- \frac{\epsilon}{4} + \frac{\sqrt{\epsilon^{2} - 8}}{4}\\1\end{matrix}\right]\right), \  \left(\lambda^+_{\pm1}= - \frac{\epsilon}{2} + \frac{\sqrt{\epsilon^{2} - 8}}{2}, v^+_{\pm1}= \left[\begin{matrix}- \frac{\epsilon}{4} - \frac{\sqrt{\epsilon^{2} - 8}}{4}\\1\end{matrix}\right]\right)\right]
\].

For the eigenvectors and eigenvalues for $(0,0)$, we can see that there is always a real positive and real negative eigenvalues for all $\epsilon$. Hence, the origin will always be a saddle. Note that $Df(0,0)$ is symmetric, implying the eigenvectors are orthogonal to each other.

For the eigenvectors and eigenvalues for $(-1,0),(1,0)$, our situation is more delicate. For $\epsilon>0$, it follows that $\lambda^-_{\pm1}<0$ and $\lambda^+_{\pm1}<0$, implying asymptotic stability. For $\epsilon=0$, we have $\lambda^-_{\pm1}=-\frac{\sqrt{-8}}{2}$ and $\lambda^+_{\pm1}=\frac{\sqrt{-8}}{2}$, meaning that linearization is inconclusive (although we can guess that it is a center). For $\epsilon<0$, it follows that $\lambda^-_{\pm1}>0$ and $\lambda^+_{\pm1}>0$, implying unstable fixed points. Also note that unlike our $(0,0)$ saddle, the eigenvectors are not orthogonal, with the dot product being 
\[ \left[\begin{matrix}- \frac{\epsilon}{4} + \frac{\sqrt{\epsilon^{2} - 8}}{4}\\1\end{matrix}\right]^\top\left[\begin{matrix}- \frac{\epsilon}{4} - \frac{\sqrt{\epsilon^{2} - 8}}{4}\\1\end{matrix}\right]=\frac{3}{2} \]

Here are the hand drawn trajectories for the local flow near the fixed points. Note that we paid attention to when the eigenvalues are complex for $(\pm 1,1)$ when $\epsilon^2-8<0$. Also note that the dominant eigenvalue for the saddle fixed point at the origin changes as $\epsilon$ changes.

\includegraphics[width=0.4\textwidth]{problem_1/sketch/off_ep_g_0_a.png}
\includegraphics[width=0.4\textwidth]{problem_1/sketch/off_ep_g_0_s.png}

\includegraphics[width=0.4\textwidth]{problem_1/sketch/off_ep_l_0_s.png}
\includegraphics[width=0.4\textwidth]{problem_1/sketch/off_ep_l_0_a.png}

\includegraphics[width=0.4\textwidth]{problem_1/sketch/on_ep_g_0.png}
\includegraphics[width=0.4\textwidth]{problem_1/sketch/on_ep_e_0.png}

\includegraphics[width=0.4\textwidth]{problem_1/sketch/on_ep_l_0.png}

Now here are the trajectories from the computer:

\includegraphics[width=\textwidth]{problem_1/figures/problem_1_a.png}

\paragraph{Part (B)}

Here is the rewrite of our system
\begin{align*}
  \dot{x}&=y\\
  \dot{y}&=-\sin x -\epsilon y^3
\end{align*}
To find the fixed points, we can set our system equal to zero getting 
\begin{align*}
  0&=\dot{x}=y\\
  0&=\dot{y}=-\sin x -\epsilon y^3=-\sin x-0=-\sin x
\end{align*}
where it becomes clear that our fixed points are $(\pm n\pi,0)$ for $n\in\mathbb{N}$. To linearize, we can compute the Jacobian 
\[ J_f(x,y) = \begin{bmatrix}
\frac{\partial f_1}{\partial x} & \frac{\partial f_1}{\partial y} \\
\frac{\partial f_2}{\partial x} & \frac{\partial f_2}{\partial y} 
\end{bmatrix} \]
with the quatities being 
\begin{gather*}
  \frac{\partial f_1}{\partial x}=0\\
  \frac{\partial f_1}{\partial y}=1\\
  \frac{\partial f_2}{\partial x}=-\cos x\\
  \frac{\partial f_2}{\partial y}=-3\epsilon y^2
\end{gather*}
Now we can compute our linearized matrices 
\begin{gather*}
  Df(0,0)=\begin{bmatrix}
    0 & 1\\
    -1 & 0
  \end{bmatrix}\\
  Df(+\pi,0)=\begin{bmatrix}
    0 & 1\\
    1 & 0
  \end{bmatrix}
\end{gather*}

Notice that I only calculated the Jacobian matrices for $0,+\pi$ since more multiples of $\pi$ will be one of these two matrices. 

Calculating the eigenvalues and eigenvectors, for equilibria $(0,0)$ and corresponding $2\pi$ multiples, we have 
\[\left[ \left( \lambda=- i, \  v= \left[\begin{matrix}i\\1\end{matrix}\right]\right), \  \left( \lambda=i, \  v=\left[\begin{matrix}- i\\1\end{matrix}\right]\right)\right]
\]
implying that linearization tells us nothing since we have 0 real part to our eigenvalues.

Calculating the eigenvalues and eigenvectors, for equilibria $(+\pi,0)$ and corresponding $2\pi$ multiples, we have 
\[\left[ \left( \lambda=-1, \  v=\left[\begin{matrix}-1\\1\end{matrix}\right]\right), \  \left( \lambda=1, \  v= \left[\begin{matrix}1\\1\end{matrix}\right]\right)\right]\] 
where we have an unstable fixed point due to the positive eigenvalue. Specifically, it is a saddle. Note that $\epsilon$ is not a factor here.

Here is the drawing for the saddle point at $(+\pi,0)$ and corresponding $2\pi$ multiples

\includegraphics[width=0.4\textwidth]{problem_1/sketch/part_b.png}

Now here are the trajectories from the computer:

\includegraphics[width=0.9\textwidth]{problem_1/figures/problem_1_b.png}

\paragraph{Part (C)}

To find the fixed points, we can set our system equal to zero getting 
\begin{align*}
  0&=\dot{x}=-x+x^2=x(x-1)\\
  0&=\dot{y}=x+y
\end{align*}
where it becomes clear that our fixed points are $(0,0)$ and $(1,-1)$. To linearize, we can compute the Jacobian
\[ J_f(x,y) = \begin{bmatrix}
\frac{\partial f_1}{\partial x} & \frac{\partial f_1}{\partial y} \\
\frac{\partial f_2}{\partial x} & \frac{\partial f_2}{\partial y} 
\end{bmatrix} \]
with the quatities being 
\begin{gather*}
  \frac{\partial f_1}{\partial x}=-1+2x\\
  \frac{\partial f_1}{\partial y}=0\\
  \frac{\partial f_2}{\partial x}=1\\
  \frac{\partial f_2}{\partial y}=1
\end{gather*}

Now we can compute our linearized matrices 
\begin{gather*}
  Df(0,0)=\begin{bmatrix}
    -1 & 0\\
    1 & 1
  \end{bmatrix}\\
  Df(1,-1)=\begin{bmatrix}
    1 & 0\\
    1 & 1
  \end{bmatrix}
\end{gather*}

Calculating the eigenvalues and eigenvectors, for equilibria $(0,0)$, we have 
\[
\left[ \left( \lambda=-1, \  v= \left[\begin{matrix}-2\\1\end{matrix}\right]\right), \  \left( \lambda=1, \  v= \left[\begin{matrix}0\\1\end{matrix}\right]\right)\right]
\]
where we have an unstable fixed point due to the positive eigenvalue. Specifically, it is a saddle.

Calculating the eigenvalues and eigenvectors, for equilibria $(1,-1)$, we have 
\[
\left[ \left( \lambda=1, \  v= \left[\begin{matrix}0\\1\end{matrix}\right]\right)\right]
\] 
where we have an unstable fixed point due to the positive eigenvalue. Note that this is an improper, or degenerate, node.

Here is the sketch for the saddle node:

\includegraphics[width=0.4\textwidth]{problem_1/sketch/part_c_saddle.png}

Here is the sketch for the improper node at $(1,-1)$. Note that we can use the equations in class that solved for the linearized trajectories around the improper node for a clue as to how to draw it 
\begin{align*}
  x(t)&=e^{\lambda t}\cdot x_0=e^{t}\\
  y(t)&=x_0te^{\lambda t}+e^{\lambda t}\cdot y_0=e^{t}=te^{ t}-e^{ t}=(t-1)e^t
\end{align*}

\includegraphics[width=0.4\textwidth]{problem_1/sketch/part_c_improper.png}

Here is the computer rendering of the system:

\includegraphics[width=0.5\textwidth]{problem_1/figures/problem_1_c.png}

\end{solution}

\clearpage

\begin{problem}
(20 points) \textit{Linear vector fields and maps.}
\begin{enumerate}
  \item[(a)] For the following linear ODE, compute the stable, unstable, and center subspaces $E^s$, $E^u$, $E^c$ of the origin. Sketch these subspaces and some representative trajectories of the system in the phase space. \[ \begin{bmatrix}
    \dot{x}_1 \\ \dot{x}_2 \\ \dot{x}_3
  \end{bmatrix} = \begin{bmatrix}
    1 & -3 & 3 \\ 3 & -5 & 3 \\ 6 & -6 & 4
  \end{bmatrix} \begin{bmatrix}
    x_1 \\ x_2 \\ x_3
  \end{bmatrix} \]
  \item[(b)] For the following linear map on $\mathbb{R}^2$, compute the stable, unstable, and center subspaces $E^s$, $E^u$, $E^c$ of the origin. What are the qualitative differences in the dynamics between $\lambda <0$, $\lambda =0$, and $\lambda >0$? Sketch the subspaces and some representative trajectories of the system in the phase space. \[ \begin{bmatrix}
    x_1 \\ x_2
  \end{bmatrix}\mapsto \begin{bmatrix}
    1 & -1 \\ 0 & \lambda
  \end{bmatrix} \begin{bmatrix}
    x_1 \\ x_2
  \end{bmatrix}, \qquad |\lambda|<1.\]
\end{enumerate}

\end{problem}

\begin{solution}
\paragraph{Part (a)}
Let's start with finding the eigenvalues. It's a slog, but we can write 
\[A-\lambda I= \begin{bmatrix}
    1 & -3 & 3 \\ 3 & -5 & 3 \\ 6 & -6 & 4
  \end{bmatrix} - \lambda I = \begin{bmatrix}
    1-\lambda & -3 & 3 \\ 3 & -5-\lambda & 3 \\ 6 & -6 & 4-\lambda
  \end{bmatrix}  \]
where the eigenvalues are computed as $\operatorname{det}\left(A-\lambda I\right)=0$. Thankfully, we can use the formula for $3\times 3$ matrix determinants of
\[ a_1(b_2c_3-b_3c_2)-a_2(b_1c_3-b_3c_1)+a_3(b_1c_2-b_2c_1) \]
Plugging everything in we get 
\begin{gather*}
  (1-\lambda)\left((-5-\lambda)(4-\lambda)-3(-6)\right)-(-3)\left(3(4-\lambda)-3\cdot 6\right)+(3)\left(3(-6)-(-5-\lambda)\cdot 6\right)\\
  \left((1-\lambda)(-5-\lambda)(4-\lambda)+18(1-\lambda)\right) +\left(9(4-\lambda)-54\right)+\left(-54-18(-5-\lambda)\right)\\
  \left((-5+4\lambda+\lambda^2)(4-\lambda)+18-18\lambda\right) +\left(36-9\lambda-54\right)+\left(-54+90+18\lambda\right)\\
  (-20+21\lambda-\lambda^3)+18-18\lambda +36-9\lambda-54-54+90+18\lambda\\
  \left(-20+18+36-54-54+90\right) + \left(21-18-9+18\right)\lambda +\lambda^3\\
  16+12\lambda+\lambda^3
\end{gather*}
Setting it equal to zero and doing some factorization, we get 
\begin{gather*}
  16+12\lambda+\lambda^3=0\\
  (4-\lambda)(4+4x+x^2)=0\\
  (4-\lambda)(2+\lambda)(2+\lambda)=0
\end{gather*}
where it follows that the eigenvalues are $\lambda_1=4$, $\lambda_2=-2$, and $\lambda_3=-2$.

To find eigenvalues, we can solve for the equation 
\[\begin{bmatrix}
    1-\lambda & -3 & 3 \\ 3 & -5-\lambda & 3 \\ 6 & -6 & 4-\lambda
  \end{bmatrix} \begin{bmatrix}
    x_1 \\ x_2 \\ x_3
  \end{bmatrix} = \begin{bmatrix}
    0 \\ 0 \\ 0
  \end{bmatrix} \]
for each eigenvalue. Starting with $\lambda_1=4$, we have 
\begin{gather*}
  \begin{bmatrix}
    1-4 & -3 & 3 \\ 3 & -5-4 & 3 \\ 6 & -6 & 4-4
  \end{bmatrix} \begin{bmatrix}
    x_1 \\ x_2 \\ x_3
  \end{bmatrix} = \begin{bmatrix}
    0 \\ 0 \\ 0
  \end{bmatrix}\\
  \begin{bmatrix}
    -3 & -3 & 3 \\ 3 & -9 & 3 \\ 6 & -6 & 0
  \end{bmatrix} \begin{bmatrix}
    x_1 \\ x_2 \\ x_3
  \end{bmatrix} = \begin{bmatrix}
    0 \\ 0 \\ 0
  \end{bmatrix}\\
\end{gather*}
where the last row implies that $x_1=x_2$. Plugging that into the expression $-3x_1-3x_2+3x_3=0$, we have 
\begin{gather*}
  -3x_1-3x_1+3x_3=0\\
  -6x_1+3x_3=0\\
  x_3=2x_1
\end{gather*}
Choosing $x_1=1$, we have an eigenvector corresponding to the eigenvalue $\lambda_1=4$ of 
\[ v_1= \begin{bmatrix}
    1 \\ 1 \\ 2
  \end{bmatrix}
\]

Moving on to the eigenvalue $\lambda_{2,3}=-2$, we can write 
\begin{gather*}
  \begin{bmatrix}
    1+2 & -3 & 3 \\ 3 & -5+2 & 3 \\ 6 & -6 & 4+2
  \end{bmatrix} \begin{bmatrix}
    x_1 \\ x_2 \\ x_3
  \end{bmatrix} = \begin{bmatrix}
    0 \\ 0 \\ 0
  \end{bmatrix}\\
  \begin{bmatrix}
    3 & -3 & 3 \\ 3 & -3 & 3 \\ 6 & -6 & 6
  \end{bmatrix} \begin{bmatrix}
    x_1 \\ x_2 \\ x_3
  \end{bmatrix} = \begin{bmatrix}
    0 \\ 0 \\ 0
  \end{bmatrix}\\
\end{gather*}
and since all rows are the same, we are justified in writing
\[x_3=-x_1+x_2\]
where the eigenvectors corresponding to $\lambda_{2,3}=-2$ span this plane. Choosing two arbitrarily, we have 
\[ v_2= \begin{bmatrix}
    0 \\ 1 \\ 1
  \end{bmatrix},\qquad v_3= \begin{bmatrix}
    -1 \\ 0 \\ 1
  \end{bmatrix}
\]

Compiling our work into the stable, unstable, and center subspaces $E^s$, $E^u$, $E^c$, we can write 
\begin{align*}
  E^u&=\operatorname{span}\left\{\begin{bmatrix}
    1 \\ 1 \\ 2
  \end{bmatrix}\right\}\\
  E^s&=\operatorname{span}\left\{\begin{bmatrix}
    0 \\ 1 \\ 1
  \end{bmatrix}, \begin{bmatrix}
    -1 \\ 0 \\ 1
  \end{bmatrix}\right\}\\
  E^c&=\operatorname{span}\left\{0\right\}
\end{align*}

Here is the requisite figure:

\includegraphics[width=0.5\textwidth]{problem_2/sketch/3_system.png}

Note the the black plane is $E^s$, the blue line is $E^u$, the red lines start just outside $E^s$ and tend towards $E^u$, and the green lines start inside $E^s$ and go to the origin.

\paragraph{Part (b)}

Okay so we can use SymPy to do the eigenvalues and eigenvectors. We have 
\[
\left[ \left( \text{eigenvalue  } 1, \  \text{eigenvector  } \left[\begin{matrix}1\\0\end{matrix}\right]\right), \  \left( \text{eigenvalue  } \lambda, \  \text{eigenvector  } \left[\begin{matrix}- \frac{1}{\lambda - 1}\\1\end{matrix}\right]\right)\right]
\]

Compiling our ``work" into the stable, unstable, and center subspaces $E^s$, $E^u$, $E^c$, we can write 
\begin{align*}
  E^s&=\operatorname{span}\left\{\begin{bmatrix}
  - \frac{1}{\lambda - 1}\\1
  \end{bmatrix}\right\}\\
  E^u&=\operatorname{span}\left\{0\right\}\\
  E^c&=\operatorname{span}\left\{\begin{bmatrix}1\\0\end{bmatrix}\right\}
\end{align*}

For $\lambda>0$ we have the following sketch of the subspaces and some representative trajectories 

\includegraphics[width=0.5\textwidth]{problem_2/sketch/problem_2_b_g.png}

For $\lambda=0$ we have the following sketch of the subspaces and some representative trajectories 

\includegraphics[width=0.5\textwidth]{problem_2/sketch/problem_2_b_e.png}

For $\lambda<0$ we have the following sketch of the subspaces and some representative trajectories 

\includegraphics[width=0.5\textwidth]{problem_2/sketch/problem_2_b_l.png}

The qualitative differences in the dyanmics for different values of $\lambda$ are as follows:
\begin{enumerate}
  \item[$\lambda>0$] --- Trajectories asymptotically decay to the center subspace with a slope equal to the stable subspace.
  \item[$\lambda=0$] --- Trajectories directly fall onto the center subspace with a slop equal to the stable subspace (slope $1$).
  \item[$\lambda<0$] --- Trajectories asymptotically decay to the center subspace with a slope equal to the stable subspace and oscillating across the x-axis, as we'd expect with negative $\lambda$ values.
\end{enumerate}

\end{solution}

\clearpage

\begin{problem}
(38 points) For the following dynamical systems, find the fixed point(s) and compute the associated stable and unstable manifolds, up to fourth order (that is, if you consider an approximation $y=h(x)$, let $h$ be a quartic polynomial). Use these results to sketch the global phase portraits.
\begin{enumerate}
  \item[(a)] For the following ODE, assume $\lambda>0$.\begin{align*}
    \dot{x}&=x(1-x^2)\\
    \dot{y}&=-\lambda(y-x^2).
  \end{align*}
  \item[(b)] For the following map, assume $|\lambda|<1$ and $|\mu|>1$.\begin{align*}
    x&\mapsto \lambda x,\\
    y&\mapsto \mu y-x^3.
  \end{align*}
\end{enumerate}

\end{problem}

\begin{solution}
\paragraph{Part (A)}
To begin, let's find the fixed points by writing 
\begin{align*}
  0&=\dot{x}=x(1-x^2)\implies x=0,x=\pm1\\
  0&=\dot{y}=-\lambda (y-x^2) \implies y=x^2
\end{align*}
Hence our fixed points are $(0,0),(-1,1),(1,1)$. To linearize, we can compute the Jacobian 
\[ J_f(x,y) = \begin{bmatrix}
\frac{\partial f_1}{\partial x} & \frac{\partial f_1}{\partial y} \\
\frac{\partial f_2}{\partial x} & \frac{\partial f_2}{\partial y} 
\end{bmatrix} \]
with the quatities being 
\begin{gather*}
  \frac{\partial f_1}{\partial x}=1-3x^2\\
  \frac{\partial f_1}{\partial y}=0\\
  \frac{\partial f_2}{\partial x}=2\lambda x\\
  \frac{\partial f_2}{\partial y}=-\lambda
\end{gather*}
Now we can compute our linearized matrices 
\begin{gather*}
  Df(0,0)=\begin{bmatrix}
    1 & 0\\
    0 & -\lambda
  \end{bmatrix}\\
  Df(-1,1)=\begin{bmatrix}
    -2 & 0\\
    -2\lambda & -\lambda
  \end{bmatrix}\\
  Df(1,1)=\begin{bmatrix}
    -2 & 0\\
    2\lambda & -\lambda
  \end{bmatrix}\\
\end{gather*}

Let's first approximate the stable and unstable manifolds associated with the $(0,0)$ fixed point. Our general workflow will be to calculate the eigenvalues and vectors, construct $h(x)$ or $g(y)$, then differentiate, group coefficients together, and solve for the coefficients.

For our fixed point $(0,0)$, we have 
\[
\left[ \left(\text{eigenvalue  }  1, \ \text{eigenvector  }   \left[\begin{matrix}1\\0\end{matrix}\right]\right), \  \left(\text{eigenvalue  }  - \lambda, \  \text{eigenvector  }  \left[\begin{matrix}0\\1\end{matrix}\right]\right)\right]
\]

Starting with our unstable manifold with the eigenvalue of $1$ and eigenvector of $\left[\begin{matrix}1\\0\end{matrix}\right]$, we can write 
\[ y=h(x)=a_0 +a_1x+a_2x^2+a_3x^3+a_4x^4 \]
From our eigenvector we can deduce that $a_0=a_1=0$. Taking the time derivative, we have 
\begin{gather*}
  \dot{y}=h'(x)=\left(2a_2x+3a_3x^2+4a_4x^3\right)\dot{x}\\
  -\lambda(y-x^2)=\left(2a_2x+3a_3x^2+4a_4x^3\right)x(1-x^2)\\
  -\lambda(a_2x^2+a_3x^3+a_4x^4-x^2)=\left(2a_2x+3a_3x^2+4a_4x^3\right)x(1-x^2)\\
\end{gather*}

I'm gonnna save some time and just parse with SymPy. Hence, we get 
\[
 \left(- a_{2} \lambda - 2 a_{2} + \lambda\right)x^{2} + \left(- a_{3} \lambda - 3 a_{3}\right) x^{3} + \left(2 a_{2} - a_{4} \lambda - 4 a_{4}\right)  x^{4} + 3 a_{3} x^{5} + 4 a_{4} x^{6} =0
\]
where we can ignore terms $3 a_{3} x^{5} + 4 a_{4} x^{6}$ since we are interested in 4th order approximations. Now we get
\begin{align*}
  a_2&=\frac{\lambda}{\lambda+2}\\
  a_3&=0 \qquad\text{since $\lambda>0$}\\
  a_4&=\frac{2a_2}{\lambda+4}
\end{align*}
Simplifying $a_4$, we get 
\begin{gather*}
  a_4=\frac{\lambda}{\lambda+2}\cdot\frac{2}{\lambda+4}\\
  a_4 = \frac{2\lambda}{(\lambda+2)(\lambda+4)}
\end{gather*}
and hence our approximated unstable manifold is 
\[ h(x)= \frac{\lambda}{\lambda+2}x^2 + \frac{2\lambda}{(\lambda+2)(\lambda+4)}x^4 \]


Moving on to the stable manifold with the eigenvalue of $-\lambda$ and eigenvector of $\left[\begin{matrix}0\\1\end{matrix}\right]$, note that for any point where $x=0$, we have the following system 
\begin{align*}
  \dot{x}&=0\\
  \dot{y}&=-\lambda y
\end{align*}
implying that trajctories that start with $x=0$ have $\lim_{t\to\infty}y(t)\to 0$, inline with the interpretation of a stable manifold. Hence, the stable manifold is simply an extension of the eigenvector and can be described by the line 
\[ x=0 \] 

Let's move on to the manifolds for $(-1,1)$. The eigenvalues and eigenvectors are 
\[
\left[ \left(\text{eigenvalue  } -2, \ \text{eigenvector  }  \left[\begin{matrix}\frac{2 - \lambda}{2 \lambda}\\1\end{matrix}\right]\right), \  \left(\text{eigenvalue  } - \lambda, \ \text{eigenvector  } \left[\begin{matrix}0\\1\end{matrix}\right]\right)\right]
\]
This is nice because both of our eigenvalues are negative and their eigenvectors are linearly independent. Hence our $E^s$ for $(-1,1)$ is all of $\mathbb{R}^2$, and the stable manifold theorem implies that our manifold approximation will be tangent to $\mathbb{R}^2$ near $(-1,1)$. Given that our system is in $\mathbb{R}^2$, an approximation of the stable manifold would be any neighborhood $U\subset \mathbb{R}^2$ as long as $U$ does not intersect $x=0$, the stable manifold of $(0,0)$. Hence, the region
\[U=\left\{ (x,y)\in\mathbb{R}^2: y\in\mathbb{R},\ x\in(-\infty,0) \right\}\]
is the stable manifold of the fixed point $(-1,1)$. There is no unstable manifold.

Moving on to the manifolds for $(1,1)$, the eigenvalues and eigenvectors are 
\[
\left[ \left( \text{eigenvalue  }-2, \  \text{eigenvector  }\left[\begin{matrix}\frac{\lambda - 2}{2 \lambda}\\1\end{matrix}\right]\right), \  \left(\text{eigenvalue  } - \lambda, \ \text{eigenvector  }\left[\begin{matrix}0\\1\end{matrix}\right]\right)\right]
\]
This is also nice because it mirrors are situation for the fixed point $(-1,1)$. Hence, our same argument applies, and the stable manifold is the region 
\[U=\left\{ (x,y)\in\mathbb{R}^2: y\in\mathbb{R},\ x\in(0,\infty) \right\}\]
where the valid region of our $x$ coordinates are fliped to accomodate our fixed point $(1,1)$.

Here is the sketch of our global phase portrait:

\includegraphics[width=0.5\textwidth]{problem_3/sketch/problem_3_a.png}

Note that we can exactly solve for the trajectories at $x=-1$ and $x=1$ since they imply 
\begin{align*}
  \dot{x}&=0\\
  \dot{y}&=-\lambda (y-1)
\end{align*}
where $y$ will simply flow to $y=1$.

\paragraph{Part (B)}

To begin, let's find the fixed points by writing 
\begin{align*}
  x&=\lambda x\\
\end{align*}
where since $|\lambda|<1$, it follows that the only fixed $x$ coord is $x=0$. For $y$, we can write 
\[ y=\mu y \]
where since $|\mu|>1$, it follows that the only fixed $y$ coord is $y=0$. Hence the only fixed point is $(0,0)$.

To linearize, we can compute the Jacobian 
\[ J_f(x,y) = \begin{bmatrix}
\frac{\partial f_1}{\partial x} & \frac{\partial f_1}{\partial y} \\
\frac{\partial f_2}{\partial x} & \frac{\partial f_2}{\partial y} 
\end{bmatrix} \]
with the quatities being 
\begin{gather*}
  \frac{\partial f_1}{\partial x}=\lambda\\
  \frac{\partial f_1}{\partial y}=0\\
  \frac{\partial f_2}{\partial x}=-3x^2\\
  \frac{\partial f_2}{\partial y}=\mu
\end{gather*}
Now we can compute our linearized matrix 
\[
Df(0,0)=\begin{bmatrix}
  \lambda & 0\\
  0 & \mu
\end{bmatrix}\\
\]
where are eigenvalues and eigenvectors are 
\[
\left[ \left( \text{eigenvalue  }\lambda, \  \text{eigenvector  }\left[\begin{matrix}1\\0\end{matrix}\right]\right), \  \left(\text{eigenvalue  } \mu, \ \text{eigenvector  }\left[\begin{matrix}0\\1\end{matrix}\right]\right)\right]
\]

Given $|\lambda|<1$, we can construct our stable manifold by writing 
\[ y=h(x)=a_0 +a_1x+a_2x^2+a_3x^3+a_4x^4 \]
where $a_0=0$ since we are at the origin and $a_1=0$ since the corresponding eigenvector is $\left[\begin{matrix}1\\0\end{matrix}\right]$. Hence we have 
\[ y=h(x)=a_2x^2+a_3x^3+a_4x^4 \]
To compute the other coefficients, we can use the following commute diagram that is implied given our map is invariant under the stable manifold:
\[
\begin{tikzcd}
x_t \arrow{r}{h} \arrow{d}{f} & y_t \arrow{d}{g} \\
x_{t+1} \arrow{r}{h} & y_{t+1}
\end{tikzcd}
\]
where $x\mapsto f(x,y)$ and $y\mapsto g(x,y)$. Hence,
\[ h(f(x,y))=g(x,h(x)) \]
and plugging in our expressions, we have 
\begin{gather*}
  h(\lambda x) = \mu\, h(x) - x^3\\
  a_2\lambda^2x^2+a_3\lambda^3x^3+a_4\lambda^4=\mu\left(a_2x^2+a_3x^3+a_4x^4\right)-x^3
\end{gather*}
where grouping terms together gives us 
\begin{align*}
  a_2&=0\\
  a_3&=\frac{1}{\mu-\lambda^3}\\
  a_4&=0
\end{align*}
because $a\lambda^n=a\mu$ implies $a=0$. Hence our stable manifold approximation is 
\[ h(x)=\frac{1}{\mu-\lambda^3}x^3 \]

Turning our attention to approximating the unstable manifold $x=h^u(y)$, we can argue that the unstable manifold is $x=0$ since our map when $x=0$ becomes 
\begin{align*}
    x&\mapsto 0,\\
    y&\mapsto \mu y - 0 = \mu y.
\end{align*}
and we merely just stay along the y-axis.

Below are the various global phase portrait sketches with example trajectories. Note that there are four conditions, one for each pairing of positive and negative $\mu$ and $\lambda$ values.

\includegraphics[width=0.45\textwidth]{problem_3/sketch/p3b_gg.png}
\includegraphics[width=0.45\textwidth]{problem_3/sketch/p3b_lg.png}

\includegraphics[width=0.45\textwidth]{problem_3/sketch/p3b_gl.png}
\includegraphics[width=0.45\textwidth]{problem_3/sketch/p3b_ll.png}


\end{solution}

\end{document}