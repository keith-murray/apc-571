\documentclass{article}
\usepackage{amsmath, amssymb}
\usepackage{mathhint}

\begin{document}

% Set the context for this problem - this information appears in the page header
% and helps the hint system understand what material you've covered
\mathhintcontext{
  book=Guckenheimer and Holmes,
  chapter=1,
  section=1.2-1.4,
  problem=1,
  lectures={lecture_3.tex, lecture_4.tex}
}

\begin{problem}
(42 points) (Modification of Ex. 1.3.2 in GH) For each of the following systems, find all of the fixed points and classify their local stability using linearization. At each fixed point, compute eigenvalues and eigenvectors of the linearization and sketch (by hand) the trajectories of the local flow near the fixed points. Then use a computer to plot sample trajectories and sketch the global phase portrait. [\textit{Hint:} Be sure to first rewrite second-order equations as first-order systems as we did in lecture, and in each case consider $\epsilon>0$, $\epsilon=0$, and $\epsilon<0$.]
\begin{enumerate}
  \item[(a)] $\ddot{x}+\epsilon\dot{x}-x+x^3=0$
  \item[(b)] $\ddot{x}+\epsilon\dot{x}^3+\sin x=0$
  \item[(c)] $\dot{x}=-x+x^2$, $\dot{y}=x+y$
\end{enumerate}

\end{problem}

\begin{notes}
\paragraph{Part (A)}
Here is the rewrite of our system 
\begin{align*}
  \dot{x}&=y\\
  \dot{y}&=x-x^3-\epsilon y
\end{align*}
To find the fixed points, we can set our system equal to zero getting 
\begin{align*}
  0&=\dot{x}=y\\
  0&=\dot{y}=x-x^3-\epsilon y=x-x^3=x(1-x^2)
\end{align*}
where it becomes clear that our fixed points are $(0,0),(-1,0),(1,0)$. To linearize, we can compute the Jacobian 
\[ J_f(x,y) = \begin{bmatrix}
\frac{\partial f_1}{\partial x} & \frac{\partial f_1}{\partial y} \\
\frac{\partial f_2}{\partial x} & \frac{\partial f_2}{\partial y} 
\end{bmatrix} \]
with the quatities being 
\begin{gather*}
  \frac{\partial f_1}{\partial x}=0\\
  \frac{\partial f_1}{\partial y}=1\\
  \frac{\partial f_2}{\partial x}=1-3x^2\\
  \frac{\partial f_2}{\partial y}=-\epsilon
\end{gather*}
Now we can compute our linearized matrices 
\begin{gather*}
  Df(0,0)=\begin{bmatrix}
    0 & 1\\
    1 & -\epsilon
  \end{bmatrix}\\
  Df(-1,0)=\begin{bmatrix}
    0 & 1\\
    -2 & -\epsilon
  \end{bmatrix}\\
  Df(1,0)=\begin{bmatrix}
    0 & 1\\
    -2 & -\epsilon
  \end{bmatrix}\\
\end{gather*}

\mathhint{Nudge}{2026-02-10 17:51}{You're on exactly the right track — the fixed points, Jacobian, and linearized matrices all look correct. Now just find the eigenvalues and eigenvectors of each matrix for the three cases ($\epsilon > 0$, $\epsilon = 0$, $\epsilon < 0$) and classify accordingly.}

Via SymPy (which we are allowed to use), we have the following eigenvalues and vectors for fixed point $(0,0)$
\[\left[ \left(\lambda^-_0= - \frac{\epsilon}{2} - \frac{\sqrt{\epsilon^{2} + 4}}{2},  \  v^-_0=\left[\begin{matrix}\frac{\epsilon}{2} - \frac{\sqrt{\epsilon^{2} + 4}}{2}\\1\end{matrix}\right]\right), \  \left(\lambda^+_0= - \frac{\epsilon}{2} + \frac{\sqrt{\epsilon^{2} + 4}}{2},  \  v^+_0=\left[ \begin{matrix}\frac{\epsilon}{2} + \frac{\sqrt{\epsilon^{2} + 4}}{2}\\1\end{matrix}\right]\right)\right]
\]
and for the $(-1,0),(1,0)$ we have 
\[
\left[ \left( \lambda^-_{\pm1}=- \frac{\epsilon}{2} - \frac{\sqrt{\epsilon^{2} - 8}}{2}, \ v^-_{\pm1} = \left[\begin{matrix}- \frac{\epsilon}{4} + \frac{\sqrt{\epsilon^{2} - 8}}{4}\\1\end{matrix}\right]\right), \  \left(\lambda^+_{\pm1}= - \frac{\epsilon}{2} + \frac{\sqrt{\epsilon^{2} - 8}}{2}, v^+_{\pm1}= \left[\begin{matrix}- \frac{\epsilon}{4} - \frac{\sqrt{\epsilon^{2} - 8}}{4}\\1\end{matrix}\right]\right)\right]
\].

For the eigenvectors and eigenvalues for $(0,0)$, we can see that there is always a real positive and real negative eigenvalues for all $\epsilon$. Hence, the origin will always be a saddle. Note that $Df(0,0)$ is symmetric, implying the eigenvectors are orthogonal to each other.

For the eigenvectors and eigenvalues for $(-1,0),(1,0)$, our situation is more delicate. For $\epsilon>0$, it follows that $\lambda^-_{\pm1}<0$ and $\lambda^+_{\pm1}<0$, implying asymptotic stability. For $\epsilon=0$, we have $\lambda^-_{\pm1}=-\frac{\sqrt{-8}}{2}$ and $\lambda^+_{\pm1}=\frac{\sqrt{-8}}{2}$, meaning that linearization is inconclusive (although we can guess that it is a center). For $\epsilon<0$, it follows that $\lambda^-_{\pm1}>0$ and $\lambda^+_{\pm1}>0$, implying unstable fixed points. Also note that unlike our $(0,0)$ saddle, the eigenvectors are not orthogonal, with the dot product being 
\[ \left[\begin{matrix}- \frac{\epsilon}{4} + \frac{\sqrt{\epsilon^{2} - 8}}{4}\\1\end{matrix}\right]^\top\left[\begin{matrix}- \frac{\epsilon}{4} - \frac{\sqrt{\epsilon^{2} - 8}}{4}\\1\end{matrix}\right]=\frac{3}{2} \]

\includegraphics[width=0.9\textwidth]{figures/problem_1_a.png}

\paragraph{Part (B)}

Here is the rewrite of our system
\begin{align*}
  \dot{x}&=y\\
  \dot{y}&=-\sin x -\epsilon y^3
\end{align*}
To find the fixed points, we can set our system equal to zero getting 
\begin{align*}
  0&=\dot{x}=y\\
  0&=\dot{y}=-\sin x -\epsilon y^3=-\sin x-0=-\sin x
\end{align*}
where it becomes clear that our fixed points are $(\pm n\pi,0)$ for $n\in\mathbb{N}$. To linearize, we can compute the Jacobian 
\[ J_f(x,y) = \begin{bmatrix}
\frac{\partial f_1}{\partial x} & \frac{\partial f_1}{\partial y} \\
\frac{\partial f_2}{\partial x} & \frac{\partial f_2}{\partial y} 
\end{bmatrix} \]
with the quatities being 
\begin{gather*}
  \frac{\partial f_1}{\partial x}=0\\
  \frac{\partial f_1}{\partial y}=1\\
  \frac{\partial f_2}{\partial x}=-\cos x\\
  \frac{\partial f_2}{\partial y}=-3\epsilon y^2
\end{gather*}
Now we can compute our linearized matrices 
\begin{gather*}
  Df(0,0)=\begin{bmatrix}
    0 & 1\\
    -1 & 0
  \end{bmatrix}\\
  Df(+\pi,0)=\begin{bmatrix}
    0 & 1\\
    1 & 0
  \end{bmatrix}
\end{gather*}

\mathhint{Nudge}{2026-02-10 21:01}{You're making great progress on part (b) — notice that the linearized matrices at the fixed points don't depend on $\epsilon$ at all, so think carefully about what that means for the classification and whether linearization alone can tell you the full story here.}

Notice that I only calculated the Jacobian matrices for $0,+\pi$ since more multiples of $\pi$ will be one of these two matrices. 

Calculating the eigenvalues and eigenvectors, for equilibria $(0,0)$ and corresponding $2\pi$ multiples, we have 
\[\left[ \left( \lambda=- i, \  v= \left[\begin{matrix}i\\1\end{matrix}\right]\right), \  \left( \lambda=i, \  v=\left[\begin{matrix}- i\\1\end{matrix}\right]\right)\right]
\]
implying that linearization tells us nothing since we have 0 real part to our eigenvalues.

Calculating the eigenvalues and eigenvectors, for equilibria $(+\pi,0)$ and corresponding $2\pi$ multiples, we have 
\[\left[ \left( \lambda=-1, \  v=\left[\begin{matrix}-1\\1\end{matrix}\right]\right), \  \left( \lambda=1, \  v= \left[\begin{matrix}1\\1\end{matrix}\right]\right)\right]\] 
where we have an unstable fixed point due to the positive eigenvalue. Specifically, it is a saddle. Note that $\epsilon$ is not a factor here.

\includegraphics[width=0.9\textwidth]{figures/problem_1_b.png}


\mathhint{Nudge}{2026-02-10 21:14}{You've done solid work on parts (a) and (b) — now it's time to tackle part (c), which is a bit different since it's already a first-order system and doesn't involve $\epsilon$ at all.}

\paragraph{Part (C)}

To find the fixed points, we can set our system equal to zero getting 
\begin{align*}
  0&=\dot{x}=-x+x^2=x(x-1)\\
  0&=\dot{y}=x+y
\end{align*}
where it becomes clear that our fixed points are $(0,0)$ and $(1,-1)$. To linearize, we can compute the Jacobian
\[ J_f(x,y) = \begin{bmatrix}
\frac{\partial f_1}{\partial x} & \frac{\partial f_1}{\partial y} \\
\frac{\partial f_2}{\partial x} & \frac{\partial f_2}{\partial y} 
\end{bmatrix} \]
with the quatities being 
\begin{gather*}
  \frac{\partial f_1}{\partial x}=-1+2x\\
  \frac{\partial f_1}{\partial y}=0\\
  \frac{\partial f_2}{\partial x}=1\\
  \frac{\partial f_2}{\partial y}=1
\end{gather*}

Now we can compute our linearized matrices 
\begin{gather*}
  Df(0,0)=\begin{bmatrix}
    -1 & 0\\
    1 & 1
  \end{bmatrix}\\
  Df(1,-1)=\begin{bmatrix}
    1 & 0\\
    1 & 1
  \end{bmatrix}
\end{gather*}

Calculating the eigenvalues and eigenvectors, for equilibria $(0,0)$, we have 
\[
\left[ \left( \lambda=-1, \  v= \left[\begin{matrix}-2\\1\end{matrix}\right]\right), \  \left( \lambda=1, \  v= \left[\begin{matrix}0\\1\end{matrix}\right]\right)\right]
\]
where we have an unstable fixed point due to the positive eigenvalue. Specifically, it is a saddle.

Calculating the eigenvalues and eigenvectors, for equilibria $(1,-1)$, we have 
\[
\left[ \left( \lambda=1, \  v= \left[\begin{matrix}0\\1\end{matrix}\right]\right)\right]
\] 
where we have an unstable fixed point due to the positive eigenvalue. Note that this is an improper, or degenerate, node.

\includegraphics[width=0.5\textwidth]{figures/problem_1_c.png}


\mathhint{Nudge}{2026-02-10 21:41}{Your work on all three parts looks solid. For part (c), double-check that you've correctly identified the algebraic and geometric multiplicity of the eigenvalue at $(1,-1)$ — the matrix is not diagonalizable there, so make sure your sketch reflects the behavior of a defective node rather than a star node.}
\mathhint{Nudge}{2026-02-12 19:27}{Your solution looks essentially complete and correct — nice work across all three parts. The only thing worth double-checking is whether you've included hand-drawn sketches near $(1,-1)$ for part (c) that clearly show the defective node structure (trajectories tangent to the single eigenvector direction).}
Hmm, any idea how I should go about drawing the defective node? I am stuck here.

\mathhint{Nudge}{2026-02-12 19:29}{Think about how the generalized eigenvector gives you a second, independent solution — as $t \to \infty$, all trajectories become tangent to the (true) eigenvector direction $\begin{pmatrix}0\\1\end{pmatrix}$.}
\end{notes}

\begin{solution}
\paragraph{Part (A)}
Here is the rewrite of our system 
\begin{align*}
  \dot{x}&=y\\
  \dot{y}&=x-x^3-\epsilon y
\end{align*}
To find the fixed points, we can set our system equal to zero getting 
\begin{align*}
  0&=\dot{x}=y\\
  0&=\dot{y}=x-x^3-\epsilon y=x-x^3=x(1-x^2)
\end{align*}
where it becomes clear that our fixed points are $(0,0),(-1,0),(1,0)$. To linearize, we can compute the Jacobian 
\[ J_f(x,y) = \begin{bmatrix}
\frac{\partial f_1}{\partial x} & \frac{\partial f_1}{\partial y} \\
\frac{\partial f_2}{\partial x} & \frac{\partial f_2}{\partial y} 
\end{bmatrix} \]
with the quatities being 
\begin{gather*}
  \frac{\partial f_1}{\partial x}=0\\
  \frac{\partial f_1}{\partial y}=1\\
  \frac{\partial f_2}{\partial x}=1-3x^2\\
  \frac{\partial f_2}{\partial y}=-\epsilon
\end{gather*}
Now we can compute our linearized matrices 
\begin{gather*}
  Df(0,0)=\begin{bmatrix}
    0 & 1\\
    1 & -\epsilon
  \end{bmatrix}\\
  Df(-1,0)=\begin{bmatrix}
    0 & 1\\
    -2 & -\epsilon
  \end{bmatrix}\\
  Df(1,0)=\begin{bmatrix}
    0 & 1\\
    -2 & -\epsilon
  \end{bmatrix}\\
\end{gather*}

Via SymPy (which we are allowed to use), we have the following eigenvalues and vectors for fixed point $(0,0)$
\[\left[ \left(\lambda^-_0= - \frac{\epsilon}{2} - \frac{\sqrt{\epsilon^{2} + 4}}{2},  \  v^-_0=\left[\begin{matrix}\frac{\epsilon}{2} - \frac{\sqrt{\epsilon^{2} + 4}}{2}\\1\end{matrix}\right]\right), \  \left(\lambda^+_0= - \frac{\epsilon}{2} + \frac{\sqrt{\epsilon^{2} + 4}}{2},  \  v^+_0=\left[ \begin{matrix}\frac{\epsilon}{2} + \frac{\sqrt{\epsilon^{2} + 4}}{2}\\1\end{matrix}\right]\right)\right]
\]
and for the $(-1,0),(1,0)$ we have 
\[
\left[ \left( \lambda^-_{\pm1}=- \frac{\epsilon}{2} - \frac{\sqrt{\epsilon^{2} - 8}}{2}, \ v^-_{\pm1} = \left[\begin{matrix}- \frac{\epsilon}{4} + \frac{\sqrt{\epsilon^{2} - 8}}{4}\\1\end{matrix}\right]\right), \  \left(\lambda^+_{\pm1}= - \frac{\epsilon}{2} + \frac{\sqrt{\epsilon^{2} - 8}}{2}, v^+_{\pm1}= \left[\begin{matrix}- \frac{\epsilon}{4} - \frac{\sqrt{\epsilon^{2} - 8}}{4}\\1\end{matrix}\right]\right)\right]
\].

For the eigenvectors and eigenvalues for $(0,0)$, we can see that there is always a real positive and real negative eigenvalues for all $\epsilon$. Hence, the origin will always be a saddle. Note that $Df(0,0)$ is symmetric, implying the eigenvectors are orthogonal to each other.

For the eigenvectors and eigenvalues for $(-1,0),(1,0)$, our situation is more delicate. For $\epsilon>0$, it follows that $\lambda^-_{\pm1}<0$ and $\lambda^+_{\pm1}<0$, implying asymptotic stability. For $\epsilon=0$, we have $\lambda^-_{\pm1}=-\frac{\sqrt{-8}}{2}$ and $\lambda^+_{\pm1}=\frac{\sqrt{-8}}{2}$, meaning that linearization is inconclusive (although we can guess that it is a center). For $\epsilon<0$, it follows that $\lambda^-_{\pm1}>0$ and $\lambda^+_{\pm1}>0$, implying unstable fixed points. Also note that unlike our $(0,0)$ saddle, the eigenvectors are not orthogonal, with the dot product being 
\[ \left[\begin{matrix}- \frac{\epsilon}{4} + \frac{\sqrt{\epsilon^{2} - 8}}{4}\\1\end{matrix}\right]^\top\left[\begin{matrix}- \frac{\epsilon}{4} - \frac{\sqrt{\epsilon^{2} - 8}}{4}\\1\end{matrix}\right]=\frac{3}{2} \]

Here are the hand drawn trajectories for the local flow near the fixed points. Note that we paid attention to when the eigenvalues are complex for $(\pm 1,1)$ when $\epsilon^2-8<0$. Also note that the dominant eigenvalue for the saddle fixed point at the origin changes as $\epsilon$ changes.

\includegraphics[width=0.4\textwidth]{sketch/off_ep_g_0_a.png}
\includegraphics[width=0.4\textwidth]{sketch/off_ep_g_0_s.png}

\includegraphics[width=0.4\textwidth]{sketch/off_ep_l_0_s.png}
\includegraphics[width=0.4\textwidth]{sketch/off_ep_l_0_a.png}

\includegraphics[width=0.4\textwidth]{sketch/on_ep_g_0.png}
\includegraphics[width=0.4\textwidth]{sketch/on_ep_e_0.png}

\includegraphics[width=0.4\textwidth]{sketch/on_ep_l_0.png}

Now here are the trajectories from the computer:

\includegraphics[width=\textwidth]{figures/problem_1_a.png}

\paragraph{Part (B)}

Here is the rewrite of our system
\begin{align*}
  \dot{x}&=y\\
  \dot{y}&=-\sin x -\epsilon y^3
\end{align*}
To find the fixed points, we can set our system equal to zero getting 
\begin{align*}
  0&=\dot{x}=y\\
  0&=\dot{y}=-\sin x -\epsilon y^3=-\sin x-0=-\sin x
\end{align*}
where it becomes clear that our fixed points are $(\pm n\pi,0)$ for $n\in\mathbb{N}$. To linearize, we can compute the Jacobian 
\[ J_f(x,y) = \begin{bmatrix}
\frac{\partial f_1}{\partial x} & \frac{\partial f_1}{\partial y} \\
\frac{\partial f_2}{\partial x} & \frac{\partial f_2}{\partial y} 
\end{bmatrix} \]
with the quatities being 
\begin{gather*}
  \frac{\partial f_1}{\partial x}=0\\
  \frac{\partial f_1}{\partial y}=1\\
  \frac{\partial f_2}{\partial x}=-\cos x\\
  \frac{\partial f_2}{\partial y}=-3\epsilon y^2
\end{gather*}
Now we can compute our linearized matrices 
\begin{gather*}
  Df(0,0)=\begin{bmatrix}
    0 & 1\\
    -1 & 0
  \end{bmatrix}\\
  Df(+\pi,0)=\begin{bmatrix}
    0 & 1\\
    1 & 0
  \end{bmatrix}
\end{gather*}

Notice that I only calculated the Jacobian matrices for $0,+\pi$ since more multiples of $\pi$ will be one of these two matrices. 

Calculating the eigenvalues and eigenvectors, for equilibria $(0,0)$ and corresponding $2\pi$ multiples, we have 
\[\left[ \left( \lambda=- i, \  v= \left[\begin{matrix}i\\1\end{matrix}\right]\right), \  \left( \lambda=i, \  v=\left[\begin{matrix}- i\\1\end{matrix}\right]\right)\right]
\]
implying that linearization tells us nothing since we have 0 real part to our eigenvalues.

Calculating the eigenvalues and eigenvectors, for equilibria $(+\pi,0)$ and corresponding $2\pi$ multiples, we have 
\[\left[ \left( \lambda=-1, \  v=\left[\begin{matrix}-1\\1\end{matrix}\right]\right), \  \left( \lambda=1, \  v= \left[\begin{matrix}1\\1\end{matrix}\right]\right)\right]\] 
where we have an unstable fixed point due to the positive eigenvalue. Specifically, it is a saddle. Note that $\epsilon$ is not a factor here.

Here is the drawing for the saddle point at $(+\pi,0)$ and corresponding $2\pi$ multiples

\includegraphics[width=0.4\textwidth]{sketch/part_b.png}

Now here are the trajectories from the computer:

\includegraphics[width=0.9\textwidth]{figures/problem_1_b.png}

\paragraph{Part (C)}

To find the fixed points, we can set our system equal to zero getting 
\begin{align*}
  0&=\dot{x}=-x+x^2=x(x-1)\\
  0&=\dot{y}=x+y
\end{align*}
where it becomes clear that our fixed points are $(0,0)$ and $(1,-1)$. To linearize, we can compute the Jacobian
\[ J_f(x,y) = \begin{bmatrix}
\frac{\partial f_1}{\partial x} & \frac{\partial f_1}{\partial y} \\
\frac{\partial f_2}{\partial x} & \frac{\partial f_2}{\partial y} 
\end{bmatrix} \]
with the quatities being 
\begin{gather*}
  \frac{\partial f_1}{\partial x}=-1+2x\\
  \frac{\partial f_1}{\partial y}=0\\
  \frac{\partial f_2}{\partial x}=1\\
  \frac{\partial f_2}{\partial y}=1
\end{gather*}

Now we can compute our linearized matrices 
\begin{gather*}
  Df(0,0)=\begin{bmatrix}
    -1 & 0\\
    1 & 1
  \end{bmatrix}\\
  Df(1,-1)=\begin{bmatrix}
    1 & 0\\
    1 & 1
  \end{bmatrix}
\end{gather*}

Calculating the eigenvalues and eigenvectors, for equilibria $(0,0)$, we have 
\[
\left[ \left( \lambda=-1, \  v= \left[\begin{matrix}-2\\1\end{matrix}\right]\right), \  \left( \lambda=1, \  v= \left[\begin{matrix}0\\1\end{matrix}\right]\right)\right]
\]
where we have an unstable fixed point due to the positive eigenvalue. Specifically, it is a saddle.

Calculating the eigenvalues and eigenvectors, for equilibria $(1,-1)$, we have 
\[
\left[ \left( \lambda=1, \  v= \left[\begin{matrix}0\\1\end{matrix}\right]\right)\right]
\] 
where we have an unstable fixed point due to the positive eigenvalue. Note that this is an improper, or degenerate, node.

Here is the sketch for the saddle node:

\includegraphics[width=0.4\textwidth]{sketch/part_c_saddle.png}

Here is the sketch for the improper node at $(1,-1)$. Note that we can use the equations in class that solved for the linearized trajectories around the improper node for a clue as to how to draw it 
\begin{align*}
  x(t)&=e^{\lambda t}\cdot x_0=e^{t}\\
  y(t)&=x_0te^{\lambda t}+e^{\lambda t}\cdot y_0=e^{t}=te^{ t}-e^{ t}=(t-1)e^t
\end{align*}

\includegraphics[width=0.4\textwidth]{sketch/part_c_improper.png}

Here is the computer rendering of the system:

\includegraphics[width=0.5\textwidth]{figures/problem_1_c.png}

\end{solution}


\end{document}
