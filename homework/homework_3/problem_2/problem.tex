\documentclass{article}
\usepackage{amsmath, amssymb}
\usepackage{mathhint}

\begin{document}

% Set the context for this problem - this information appears in the page header
% and helps the hint system understand what material you've covered
\mathhintcontext{
  book=Guckenheimer and Holmes,
  chapter=1,
  section=1.5-1.8,
  problem=2,
  lectures={lecture_5.tex, lecture_6.tex}
}

\begin{problem}
We consider the ``rabbit vs. sheep'' system described by the following equations:
\begin{align*}
  \dot{x}&=x(3-x-2y)\\
  \dot{y}&=y(2-x-y),
\end{align*}
where $x$ represents the population of rabbits, and $y$ represents the population of sheep.
\begin{itemize}
  \item[(a)] Find all the equilibria of the system, and for each determine its index.
  \item[(b)] Show that it is impossible to have a closed orbit for the system (index theory will help but you will need something else to rule out all possible closed orbit).
\end{itemize}

\end{problem}

\begin{notes}
From office hours, I got the following hint about part (b): If you start with no rabbits, you will never have any rabbits. It you start with no sheep, you will never have any sheep. Think about invariant sets in the phase portrait. The invariant sets are the x and y axes. You cannot have an closed orbit that crosses these axes or you violate the existence and uniqueness of solutions.

\section{Part (a)}

Finding the fixed points, we set the system to zero to get 
\begin{align*}
  0&=\dot{x}=x(3-x-2y)\\
  0&=\dot{y}=y(2-x-y)
\end{align*}
Clearly, fixed points arise when $x=0$ and $y=0$. In particular, a fixed point at the origin $(0,0)$, when $x=0$ we have 
\[0=2-0-y\implies y=2\]
giving a fixed point at $(0,2)$, and when $y=0$ we have 
\[0=3-x-2(0)\implies x=3\]
giving a fixed point at $(3,0)$. When $x\neq0$ and $y\neq0$, we can write 
\begin{align*}
  0&=3-x-2y\\
  0&=2-x-y
\end{align*}
and then 
\begin{gather*}
  3-x-2y=2-x-y\\
  -y=-1\\
  y=1
\end{gather*}
Plugging $y=1$ into $0=2-x-y$, we get a fixed point at $(1,1)$.

Hence, all our fixed points are 
\begin{itemize}
  \item $(0,0)$
  \item $(0,2)$
  \item $(3,0)$
  \item $(1,1)$
\end{itemize}
Now let's linearize. We can generically write the Jacobian as 
\[ J_f(x,y) = \begin{bmatrix}
3-2x-2y & -2x \\
-y & 2-x-2y
\end{bmatrix} \]
Now let's compute the eigenvalues for each Jacobian around each fixed point.

For $(0,0)$, we have 
\[ J_f(0,0) = \begin{bmatrix}
3 & 0\\
0 & 2
\end{bmatrix} \]
where the eigenvalues, found via the diagonal, are $\lambda_0=3$ and $\lambda_1=2$. Hence, this is an unstable fixed point (source), with an index of $+1$.

For $(0,2)$, we have 
\[ J_f(0,2) = \begin{bmatrix}
-1 & 0\\
-2 & -2
\end{bmatrix} \]
where the eigenvalues, found via the diagonal, are $\lambda_0=-1$ and $\lambda_1=-2$. Hence, this is a stable fixed point (sink), with an index of $+1$.

For $(3,0)$, we have 
\[ J_f(3,0) = \begin{bmatrix}
-3 & -6\\
0 & -1
\end{bmatrix} \]
where the eigenvalues, found via the diagonal, are $\lambda_0=-3$ and $\lambda_1=-1$. Hence, this is a stable fixed point (sink), with an index of $+1$.

For $(1,1)$, we have 
\[ J_f(3,0) = \begin{bmatrix}
-1 & -2\\
-1 & -1
\end{bmatrix} \]
where the eigenvalues, found via SymPy, are $\lambda_{0,1}=-1\pm\sqrt{2}$, where $-1+\sqrt{2}$ is positive. Hence, there is are positive and negative eigenvalues, implying the fixed point is a saddle, with an index of $-1$.

\mathhint{Nudge}{2026-02-18 22:22}{You're doing great on part (a) — all equilibria and their indices look correct. For part (b), you already have a strong hint about the axes being invariant; now think about how that constraint, combined with what index theory tells you about which equilibria could be enclosed by a closed orbit, rules out every possibility.}

\section{Part (b)}

To start, let's note the following corollary on page 51 of the textbook:

\noindent
\textbf{Corollary 1.8.5.} \textit{Inside any closed orbit $\gamma$ there must be at least one fixed point. If all the fixed points within $\gamma$ are hyperbolic, then there must be an odd number, $2n+1$, of which $n$ are saddles and $n+1$ either sinks or sources.}

Looking at our system, all four fixed points are hyperbolic. Hence, if there is any closed orbit, it must contain the one saddle at $(1,1)$ and 
\begin{itemize}
  \item the two sinks $(0,2),(3,0)$,
  \item the $(0,2)$ sink and the $(0,0)$ source,
  \item or the $(3,0)$ sink and the $(0,0)$ source.
\end{itemize}

What's nice about our sinks and sources is that they all lie along the x or y axis. From our system of equations, notice that if $x=0$, then it follows that 
\begin{align*}
  \dot{x}&=0\\
  \dot{y}&=y(2-y),
\end{align*}
which means that trajectories starting on the y-axis stay on the y-axis, implying that the y-axis is an invariant set. Also notice that if $y=0$, then it follows that 
\begin{align*}
  \dot{x}&=x(3-x)\\
  \dot{y}&=0,
\end{align*}
which means that trajectories starting on the x-axis stay on the x-axis, implying that the x-axis is an invariant set. Hence, if a closed orbit were to encompass one of our sinks or sources, the closed orbit would have to cross the x or y axis, both being invariant sets. This is impossible since we would violate the existence and uniqueness of solutions at the point the closed orbit intersects the axis. Therefore, any closed orbit in the system cannot contain $(0,2),(3,0),(0,0)$, and via our reasoning with index theory, it is impossible to have a closed orbit for the system.

\mathhint{Nudge}{2026-02-19 00:35}{Your argument is essentially complete and well-structured. Just tighten up one small point: make explicit *why* a closed orbit crossing an invariant axis violates uniqueness (a point on the axis would belong to two distinct trajectories — the closed orbit and the axis trajectory — contradicting uniqueness).}
Sure. How about:

Hence, if a closed orbit were to encompass one of our sinks or sources, the closed orbit would have to cross the x or y axis, both being invariant sets. This is impossible since we would violate the existence and uniqueness of solutions at the point $p$ the closed orbit intersects the axis since $p$ would belong to two distinct trajectories. Therefore, any closed orbit in the system cannot contain $(0,2),(3,0),(0,0)$, and via our reasoning with index theory, it is impossible to have a closed orbit for the system.

\mathhint{Nudge}{2026-02-19 00:36}{Your revised wording is clear and correct — nicely done. You're ready to write up the final clean version.}
\end{notes}

\begin{solution}
\section{Part (a)}

Finding the fixed points, we set the system to zero to get 
\begin{align*}
  0&=\dot{x}=x(3-x-2y)\\
  0&=\dot{y}=y(2-x-y)
\end{align*}
Clearly, fixed points arise when $x=0$ and $y=0$. In particular, a fixed point at the origin $(0,0)$, when $x=0$ we have 
\[0=2-0-y\implies y=2\]
giving a fixed point at $(0,2)$, and when $y=0$ we have 
\[0=3-x-2(0)\implies x=3\]
giving a fixed point at $(3,0)$. When $x\neq0$ and $y\neq0$, we can write 
\begin{align*}
  0&=3-x-2y\\
  0&=2-x-y
\end{align*}
and then 
\begin{gather*}
  3-x-2y=2-x-y\\
  -y=-1\\
  y=1
\end{gather*}
Plugging $y=1$ into $0=2-x-y$, we get a fixed point at $(1,1)$.

Hence, all our fixed points are 
\begin{itemize}
  \item $(0,0)$
  \item $(0,2)$
  \item $(3,0)$
  \item $(1,1)$
\end{itemize}
Now let's linearize. We can generically write the Jacobian as 
\[ J_f(x,y) = \begin{bmatrix}
3-2x-2y & -2x \\
-y & 2-x-2y
\end{bmatrix} \]
Now let's compute the eigenvalues for each Jacobian around each fixed point.

For $(0,0)$, we have 
\[ J_f(0,0) = \begin{bmatrix}
3 & 0\\
0 & 2
\end{bmatrix} \]
where the eigenvalues, found via the diagonal, are $\lambda_0=3$ and $\lambda_1=2$. Hence, this is an unstable fixed point (source), with an index of $+1$.

For $(0,2)$, we have 
\[ J_f(0,2) = \begin{bmatrix}
-1 & 0\\
-2 & -2
\end{bmatrix} \]
where the eigenvalues, found via the diagonal, are $\lambda_0=-1$ and $\lambda_1=-2$. Hence, this is a stable fixed point (sink), with an index of $+1$.

For $(3,0)$, we have 
\[ J_f(3,0) = \begin{bmatrix}
-3 & -6\\
0 & -1
\end{bmatrix} \]
where the eigenvalues, found via the diagonal, are $\lambda_0=-3$ and $\lambda_1=-1$. Hence, this is a stable fixed point (sink), with an index of $+1$.

For $(1,1)$, we have 
\[ J_f(3,0) = \begin{bmatrix}
-1 & -2\\
-1 & -1
\end{bmatrix} \]
where the eigenvalues, found via SymPy, are $\lambda_{0,1}=-1\pm\sqrt{2}$, where $-1+\sqrt{2}$ is positive. Hence, there is are positive and negative eigenvalues, implying the fixed point is a saddle, with an index of $-1$.

\section{Part (b)}

To start, let's note the following corollary on page 51 of the textbook:

\noindent
\textbf{Corollary 1.8.5.} \textit{Inside any closed orbit $\gamma$ there must be at least one fixed point. If all the fixed points within $\gamma$ are hyperbolic, then there must be an odd number, $2n+1$, of which $n$ are saddles and $n+1$ either sinks or sources.}

Looking at our system, all four fixed points are hyperbolic. Hence, if there is any closed orbit, it must contain the one saddle at $(1,1)$ and 
\begin{itemize}
  \item the two sinks $(0,2),(3,0)$,
  \item the $(0,2)$ sink and the $(0,0)$ source,
  \item or the $(3,0)$ sink and the $(0,0)$ source.
\end{itemize}

What's nice about our sinks and sources is that they all lie along the x or y axis. From our system of equations, notice that if $x=0$, then it follows that 
\begin{align*}
  \dot{x}&=0\\
  \dot{y}&=y(2-y),
\end{align*}
which means that trajectories starting on the y-axis stay on the y-axis, implying that the y-axis is an invariant set. Also notice that if $y=0$, then it follows that 
\begin{align*}
  \dot{x}&=x(3-x)\\
  \dot{y}&=0,
\end{align*}
which means that trajectories starting on the x-axis stay on the x-axis, implying that the x-axis is an invariant set.

Hence, if a closed orbit were to encompass one of our sinks or sources, the closed orbit would have to cross the x or y axis, both being invariant sets. This is impossible since we would violate the existence and uniqueness of solutions at some point $p$ where closed orbit intersects the axis since $p$ would belong to two distinct trajectories. Therefore, any closed orbit in the system cannot contain $(0,2),(3,0),(0,0)$, and via our reasoning with index theory, it is impossible to have a closed orbit for the system.

\end{solution}

\end{document}
