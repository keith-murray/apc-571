\documentclass{article}
\usepackage{amsmath, amssymb}
\usepackage{mathhint}

\begin{document}

% Set the context for this problem - this information appears in the page header
% and helps the hint system understand what material you've covered
\mathhintcontext{
  book=Guckenheimer and Holmes,
  chapter=1,
  section=1.0-1.1,
  problem=2,
  lectures={lecture_1.tex, lecture_2.tex}
}

\begin{problem}
(10 points) Consider the differential equation
\[ \dot{x}=x^{1/3}, \qquad x(0)=0.\]
The origin $x=0$ is an equilibrium point, so one solution is $x(t)=0$.
\begin{enumerate}
  \item[(a)] Find another solution that satisfies $x(0)=0$, but $x(t)\neq 0$ for $t>0$, and therefore conclude that the solution is not unique. (Why does the uniqueness theorem not apply?)
  \item[(b)] Find an infinite family of solutions that satisy the equation with $x(0)=0$.
\end{enumerate}

\end{problem}

\begin{notes}
\paragraph{Part (a)}
Let's separate variables and solve the ODE:
\begin{gather*}
  \int x^{-1/3}dx=\int dt\\ 
  \frac{3}{2}x^{2/3}=t+C\\
  x(t)=\left( \frac{2}{3}\left(t+C\right) \right)^{3/2}
\end{gather*}
For the condition $x(0)=0$, it follows that $C=0$ and we have
\[ x(t)=\left( \frac{2}{3}t \right)^{3/2}.\]
The uniqueness theorem doesn't apply because the derivative of $x^{1/3}$ at $x=0$ is undefined since
\[f'(x)=\frac{1}{3x^{2/3}}. \]
(Local) uniqueness of solutions only apply for Lipschitz continuous functions, and continuously differentiable functions are synonmous with Lipschitz continuity.

\paragraph{Part (b)}
We can make an infinite family of solutions be defining the following piecewise equation
\[
x(t) = \begin{cases}
    0, & t<t_0 \\
    \left(\frac{2}{3}(t-t_0)\right)^{3/2}, & t \geq t_0
\end{cases}
\]
where we have $x(t_0)=0$ and 
\begin{align*}
  \frac{d}{dt}\left[\left(\frac{2}{3}(t-t_0)\right)^{3/2}\right]&=\frac{3}{2}\left(\frac{2}{3}(t-t_0)\right)^{1/2}\cdot \frac{d}{dt}\left[ \frac{2}{3}(t-t_0) \right]\\
  &=\frac{3}{2}\left(\frac{2}{3}(t-t_0)\right)^{1/2}\cdot \frac{2}{3}\\
  &=\left(\left(\frac{2}{3}(t-t_0)\right)^{3/2}\right)^{1/3}\\
  &=x^{1/3}
\end{align*}
and hence, we still have $\dot{x}=x^{1/3}$.

Given that $t_0$ is arbitrary, our solution 
\[
x(t) = \begin{cases}
    0, & t<t_0 \\
    \left(\frac{2}{3}(t-t_0)\right)^{3/2}, & t \geq t_0
\end{cases}
\]
yields and infinite number of solutions.
\mathhint{Nudge}{2026-02-02 12:23}{Your approach and solutions for both parts are excellent and mathematically sound! You've correctly identified the key issue with the uniqueness theorem and constructed the infinite family of solutions perfectly. 

One small note: you might want to double-check your statement about continuously differentiable functions being synonymous with Lipschitz continuity - while $C^1$ functions are locally Lipschitz, the converse isn't always true, but your main point about why uniqueness fails here is absolutely correct.}

\end{notes}

\begin{solution}
\paragraph{Part (a)}
Let's separate variables and solve the ODE:
\begin{gather*}
  \int x^{-1/3}dx=\int dt\\ 
  \frac{3}{2}x^{2/3}=t+C\\
  x(t)=\left( \frac{2}{3}\left(t+C\right) \right)^{3/2}
\end{gather*}
For the condition $x(0)=0$, it follows that $C=0$ and we have
\[ x(t)=\left( \frac{2}{3}t \right)^{3/2}.\]
The uniqueness theorem doesn't apply because the derivative of $x^{1/3}$ at $x=0$ is undefined since
\[f'(x)=\frac{1}{3x^{2/3}}. \]
(Local) uniqueness of solutions only apply for (locally) Lipschitz continuous functions, and (locally) continuously differentiable functions imply (local) Lipschitz continuity.

\paragraph{Part (b)}
We can make an infinite family of solutions be defining the following piecewise equation
\[
x(t) = \begin{cases}
    0, & t<t_0 \\
    \left(\frac{2}{3}(t-t_0)\right)^{3/2}, & t \geq t_0
\end{cases}
\]
where we have $x(t_0)=0$ and 
\begin{align*}
  \frac{d}{dt}\left[\left(\frac{2}{3}(t-t_0)\right)^{3/2}\right]&=\frac{3}{2}\left(\frac{2}{3}(t-t_0)\right)^{1/2}\cdot \frac{d}{dt}\left[ \frac{2}{3}(t-t_0) \right]\\
  &=\frac{3}{2}\left(\frac{2}{3}(t-t_0)\right)^{1/2}\cdot \frac{2}{3}\\
  &=\left(\left(\frac{2}{3}(t-t_0)\right)^{3/2}\right)^{1/3}\\
  &=x^{1/3}
\end{align*}
and hence, we still have $\dot{x}=x^{1/3}$.

Given that $t_0$ is arbitrary, our solution 
\[
x(t) = \begin{cases}
    0, & t<t_0 \\
    \left(\frac{2}{3}(t-t_0)\right)^{3/2}, & t \geq t_0
\end{cases}
\]
yields and infinite number of solutions.

\mathsolutioncheck{Correct}{2026-02-02 12:26}{The solution correctly finds the non-trivial solution using separation of variables, properly identifies why uniqueness fails ($f'(x)$ is unbounded at $x=0$), and constructs a valid infinite family of piecewise solutions with arbitrary switching times $t_0$.}

\end{solution}

\end{document}
