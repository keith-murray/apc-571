\documentclass{article}
\usepackage{amsmath, amssymb}
\usepackage{mathhint}

\begin{document}

% Set the context for this problem - this information appears in the page header
% and helps the hint system understand what material you've covered
\mathhintcontext{
  book=Guckenheimer and Holmes,
  chapter=1,
  section=1.0-1.1,
  problem=3,
  lectures={lecture_1.tex, lecture_2.tex}
}

\begin{problem}
(20 points) Consider the system of ODEs given by
\begin{align*}
  \dot{x}&=-y+ax(x^2+y^2)\\
  \dot{y}&=x+ay(x^2+y^2)
\end{align*}
\begin{enumerate}
  \item[(a)] Find all the equilibrium points for this system (for any value of the parameter $a$).
  \item[(b)] Find the equations linearized about $(x,y)=(0,0)$. What does the linearization let you conclude about the stability of the equilibrium point (for the nonlinear system)?
  \item[(c)] Write the system (1) in polar coordinates $(r,\theta)$, with \begin{gather*}
    x=r\cos\theta\\
    y=r\sin\theta
  \end{gather*} and determine the stability type of the equilibrium at the origin. Consider separately the cases $\alpha<0$, $\alpha=0$, and $\alpha>0$.
\end{enumerate}

\end{problem}

\begin{notes}
\paragraph{Part (a)}
Let's start by setting our system equal to zero
\begin{align}
  0&=-y+ax(x^2+y^2)\\
  0&=x+ay(x^2+y^2)
\end{align}
Now multiply equation (1) by $x$ and equation (2) by $y$ to get 
\begin{align*}
  0&=-xy+ax^2(x^2+y^2)\\
  0&=xy+ay^2(x^2+y^2)
\end{align*}
and adding them gives us 
\begin{align*}
  0&=-xy+ax^2(x^2+y^2)+xy+ay^2(x^2+y^2)\\
  &=ax^2(x^2+y^2)+ay^2(x^2+y^2)\\
  &=a(x^2+y^2)(x^2+y^2)\\
  &=a(x^2+y^2)^2
\end{align*}
where if $a\neq 0$, then our only fixed point is the origin $(0,0)$. If $a=0$, then our earlier equations are $\dot{x}=-y$ and $\dot{y}=x$ and the origin is still our only fixed point.

\mathhint{Nudge}{2026-02-03 21:48}{Your approach for part (a) is on the right track! You correctly identified that the origin $(0,0)$ is always an equilibrium point, and your algebraic manipulation is sound. However, you might want to double-check whether you've found *all* possible equilibrium points by looking more carefully at what happens when $x^2 + y^2 = 0$ versus when $x^2 + y^2 \neq 0$.}

\paragraph{Part (b)}
Our linearization becomes $\dot{\xi}=Df(x_0)\xi$ where $Df(x_0)$ is the jacobian
\[ J_f(x,y) = \begin{bmatrix}
\frac{\partial f_1}{\partial x} & \frac{\partial f_1}{\partial y} \\
\frac{\partial f_2}{\partial x} & \frac{\partial f_2}{\partial y} 
\end{bmatrix} \]
Evaluating each expression, we have 
\begin{gather*}
  \frac{\partial f_1}{\partial x}=a(x^2+y^2)+ax(2x)=ay^2+3ax^3\\
  \frac{\partial f_1}{\partial y}=-1+2axy\\
  \frac{\partial f_2}{\partial x}=1+2axy\\
  \frac{\partial f_2}{\partial y}=a(x^2+y^2)+ay(2y)=ax^2+3ay^3
\end{gather*}
Evaluating each of the expressions at $(0,0)$, we have 
\[Df(x_0)=\begin{bmatrix}
0 & -1 \\
1 & 0
\end{bmatrix} \]
and our linearization is 
\[\dot{\xi}=\begin{bmatrix}
0 & -1 \\
1 & 0
\end{bmatrix}\xi\]
To determine stability, we can compute the eigenvalues via $\det{(A-\lambda I)}=0$.
\[0=\det{(A-\lambda I)}=\lambda^2+1\]
implying that $\lambda=\pm i$ and that we cannot conclude anything about the stability of the equilibrium point.

\paragraph{Part (c)}
Let's start with deriving $\dot{r}$ via the relationship $r^2=x^2+y^2$. We can write 
\begin{gather*}
  \frac{d}{dt}\left[r^2=x^2+y^2\right]\\
  2r\dot{r}=2x\dot{x}+2y\dot{y}
\end{gather*}
and substitute in the values for $\dot{x},\dot{y}$ to get 
\begin{gather*}
  2r\dot{r}=2x\left(-y+ax(x^2+y^2)\right)+2y\left(x+ay(x^2+y^2)\right)\\
  2r\dot{r}=-2xy+2ax^2(x^2+y^2)+2xy+2ay^2(x^2+y^2)\\
  2r\dot{r}=2ax^2(x^2+y^2)+2ay^2(x^2+y^2)\\
  2r\dot{r}=2a(x^2+y^2)(x^2+y^2)\\
\end{gather*}
and substituting in $r^2=x^2+y^2$, we can write 
\begin{gather*}
  2r\dot{r}=2ar^4\\
  \dot{r}=ar^3
\end{gather*}

To derive $\dot{\theta}$, let's differentiate the relationship $\tan\theta=y/x$ to get 
\[ 
  \sec^2\theta \cdot\dot{\theta}=\frac{x\dot{y}-y\dot{x}}{x^2}\\
\]
and plugging in $x^2=r^2\cos^2\theta $, we can write 
\begin{gather*}
  \sec^2\theta \cdot\dot{\theta}=\frac{x\dot{y}-y\dot{x}}{r^2\cos^2\theta}\\
  \sec^2\theta \cdot\dot{\theta}=\frac{x\dot{y}-y\dot{x}}{r^2}\sec^2\theta\\
  \dot{\theta}=\frac{x\dot{y}-y\dot{x}}{r^2}
\end{gather*}
Substituting for $\dot{x},\dot{y}$, we have 
\begin{gather*}
  \dot{\theta}=\frac{x\left(x+ay(x^2+y^2)\right)-y\left(-y+ax(x^2+y^2)\right)}{r^2}\\
  \dot{\theta}=\frac{x^2+axy(x^2+y^2)+y^2-axy(x^2+y^2)}{r^2}\\
  \dot{\theta}=\frac{x^2+y^2}{r^2}\\
  \dot{\theta}=\frac{r^2}{r^2}\\
  \dot{\theta}=1
\end{gather*}

Hence, system (1) in polar coordinates is 
\begin{align*}
  \dot{r}&=ar^3\\
  \dot{\theta}&=1
\end{align*}
To determine the stability type of the equilibrium at the origin, let's consider separately the cases $\alpha<0$, $\alpha=0$, and $\alpha>0$.

For $\alpha<0$, we have $\dot{r}<0$, implying that 
\[||(x(t),y(t))-(0,0)||= \sqrt{x(t)^2+y(t)^2}=\sqrt{r(t)}<\sqrt{r(0)}= \sqrt{x(0)^2+y(0)^2}=||(x(0),y(0))-(0,0)||\]
Hence, given some $\epsilon>0$, we can set $\delta=\epsilon$ and it follows that for $||(x(0),y(0))-(0,0)||<\delta$ we have 
\[||(x(t),y(t))-(0,0)||< ||(x(0),y(0))-(0,0)||<\delta=\epsilon\]
and 
\[ ||(x(t),y(t))-(0,0)|| < \epsilon.\] 
Furthermore, our reasoning above implies that our trajectories are strictly decreasing and bounded below by the origin. Hence, via the monotone convergence theorem, we can state the as $t\to\infty$ we have $(x(t),y(t))\to (0,0)$ and the origin is asymptotically stable.

For $\alpha=0$, we have $\dot{r}=0$ implying that $r(0)=r(t)$. Hence, we can write 
\[||(x(t),y(t))-(0,0)||= \sqrt{x(t)^2+y(t)^2}=\sqrt{r(t)}=\sqrt{r(0)}= \sqrt{x(0)^2+y(0)^2}=||(x(0),y(0))-(0,0)||\]
Given some $\epsilon>0$, we can set $\delta=\epsilon$ and it follows that for $||(x(0),y(0))-(0,0)||<\delta$ we have 
\[||(x(t),y(t))-(0,0)||=||(x(0),y(0))-(0,0)||<\delta=\epsilon\]
and
\[ ||(x(t),y(t))-(0,0)|| < \epsilon.\] 
Thus, our system is Lyapunov stable.

For $\alpha>0$, we can solve for $r$ to show that there are initial conditions that result in finite time blow up, showing the origin is unstable. By separating variables, we can write 
\begin{gather*}
  \frac{dr}{ar^3}=dt\\
  \int \frac{dr}{ar^3}=\int dt\\
  -\frac{1}{2ar^2}=t+C\\
  -\frac{1}{2ar^2}=t+C\\
  r(t)=\frac{1}{\sqrt{-2at+D}}
\end{gather*}
For the initial condition $r(0)=1$, we have $D=1$ and the resulting equation 
\[ r(t)=\frac{1}{\sqrt{-2at+1}} \]
experiences finite time blow up as $t\to 1$ given that $a>0$. Hence the origin is unstable.

\end{notes}

\begin{solution}
\paragraph{Part (a)}
Let's start by setting our system equal to zero
\begin{align}
  0&=-y+ax(x^2+y^2)\\
  0&=x+ay(x^2+y^2)
\end{align}
Now multiply equation (1) by $x$ and equation (2) by $y$ to get 
\begin{align*}
  0&=-xy+ax^2(x^2+y^2)\\
  0&=xy+ay^2(x^2+y^2)
\end{align*}
and adding them gives us 
\begin{align*}
  0&=-xy+ax^2(x^2+y^2)+xy+ay^2(x^2+y^2)\\
  &=ax^2(x^2+y^2)+ay^2(x^2+y^2)\\
  &=a(x^2+y^2)(x^2+y^2)\\
  &=a(x^2+y^2)^2
\end{align*}
where if $a\neq 0$, then our only fixed point is the origin $(0,0)$. If $a=0$, then our earlier equations are $\dot{x}=-y$ and $\dot{y}=x$ and the origin is still our only fixed point.


\paragraph{Part (b)}
Our linearization becomes $\dot{\xi}=Df(x_0)\xi$ where $Df(x_0)$ is the jacobian
\[ J_f(x,y) = \begin{bmatrix}
\frac{\partial f_1}{\partial x} & \frac{\partial f_1}{\partial y} \\
\frac{\partial f_2}{\partial x} & \frac{\partial f_2}{\partial y} 
\end{bmatrix} \]
Evaluating each expression, we have 
\begin{gather*}
  \frac{\partial f_1}{\partial x}=a(x^2+y^2)+ax(2x)=ay^2+3ax^3\\
  \frac{\partial f_1}{\partial y}=-1+2axy\\
  \frac{\partial f_2}{\partial x}=1+2axy\\
  \frac{\partial f_2}{\partial y}=a(x^2+y^2)+ay(2y)=ax^2+3ay^3
\end{gather*}
Evaluating each of the expressions at $(0,0)$, we have 
\[Df(x_0)=\begin{bmatrix}
0 & -1 \\
1 & 0
\end{bmatrix} \]
and our linearization is 
\[\dot{\xi}=\begin{bmatrix}
0 & -1 \\
1 & 0
\end{bmatrix}\xi\]
To determine stability, we can compute the eigenvalues via $\det{(A-\lambda I)}=0$.
\[0=\det{(A-\lambda I)}=\lambda^2+1\]
implying that $\lambda=\pm i$ and that we cannot conclude anything about the stability of the equilibrium point.

\paragraph{Part (c)}
Let's start with deriving $\dot{r}$ via the relationship $r^2=x^2+y^2$. We can write 
\begin{gather*}
  \frac{d}{dt}\left[r^2=x^2+y^2\right]\\
  2r\dot{r}=2x\dot{x}+2y\dot{y}
\end{gather*}
and substitute in the values for $\dot{x},\dot{y}$ to get 
\begin{gather*}
  2r\dot{r}=2x\left(-y+ax(x^2+y^2)\right)+2y\left(x+ay(x^2+y^2)\right)\\
  2r\dot{r}=-2xy+2ax^2(x^2+y^2)+2xy+2ay^2(x^2+y^2)\\
  2r\dot{r}=2ax^2(x^2+y^2)+2ay^2(x^2+y^2)\\
  2r\dot{r}=2a(x^2+y^2)(x^2+y^2)\\
\end{gather*}
and substituting in $r^2=x^2+y^2$, we can write 
\begin{gather*}
  2r\dot{r}=2ar^4\\
  \dot{r}=ar^3
\end{gather*}

To derive $\dot{\theta}$, let's differentiate the relationship $\tan\theta=y/x$ to get 
\[ 
  \sec^2\theta \cdot\dot{\theta}=\frac{x\dot{y}-y\dot{x}}{x^2}\\
\]
and plugging in $x^2=r^2\cos^2\theta $, we can write 
\begin{gather*}
  \sec^2\theta \cdot\dot{\theta}=\frac{x\dot{y}-y\dot{x}}{r^2\cos^2\theta}\\
  \sec^2\theta \cdot\dot{\theta}=\frac{x\dot{y}-y\dot{x}}{r^2}\sec^2\theta\\
  \dot{\theta}=\frac{x\dot{y}-y\dot{x}}{r^2}
\end{gather*}
Substituting for $\dot{x},\dot{y}$, we have 
\begin{gather*}
  \dot{\theta}=\frac{x\left(x+ay(x^2+y^2)\right)-y\left(-y+ax(x^2+y^2)\right)}{r^2}\\
  \dot{\theta}=\frac{x^2+axy(x^2+y^2)+y^2-axy(x^2+y^2)}{r^2}\\
  \dot{\theta}=\frac{x^2+y^2}{r^2}\\
  \dot{\theta}=\frac{r^2}{r^2}\\
  \dot{\theta}=1
\end{gather*}

Hence, system (1) in polar coordinates is 
\begin{align*}
  \dot{r}&=ar^3\\
  \dot{\theta}&=1
\end{align*}
To determine the stability type of the equilibrium at the origin, let's consider separately the cases $\alpha<0$, $\alpha=0$, and $\alpha>0$.

For $\alpha<0$, we have $\dot{r}<0$, implying that 
\[||(x(t),y(t))-(0,0)||= \sqrt{x(t)^2+y(t)^2}=r(t)<r(0)= \sqrt{x(0)^2+y(0)^2}=||(x(0),y(0))-(0,0)||\]
Hence, given some $\epsilon>0$, we can set $\delta=\epsilon$ and it follows that for $||(x(0),y(0))-(0,0)||<\delta$ we have 
\[||(x(t),y(t))-(0,0)||< ||(x(0),y(0))-(0,0)||<\delta=\epsilon\]
and 
\[ ||(x(t),y(t))-(0,0)|| < \epsilon.\] 
Furthermore, our reasoning above implies that our trajectories are strictly decreasing and bounded below by the origin. Hence, via the monotone convergence theorem, we can state the as $t\to\infty$ we have $(x(t),y(t))\to (0,0)$ and the origin is asymptotically stable.

For $\alpha=0$, we have $\dot{r}=0$ implying that $r(0)=r(t)$. Hence, we can write 
\[||(x(t),y(t))-(0,0)||= \sqrt{x(t)^2+y(t)^2}=r(t)=r(0)= \sqrt{x(0)^2+y(0)^2}=||(x(0),y(0))-(0,0)||\]
Given some $\epsilon>0$, we can set $\delta=\epsilon$ and it follows that for $||(x(0),y(0))-(0,0)||<\delta$ we have 
\[||(x(t),y(t))-(0,0)||=||(x(0),y(0))-(0,0)||<\delta=\epsilon\]
and
\[ ||(x(t),y(t))-(0,0)|| < \epsilon.\] 
Thus, our system is Lyapunov stable.

For $\alpha>0$, we can solve for $r$ to show that there are initial conditions that result in finite time blow up, showing the origin is unstable. By separating variables, we can write 
\begin{gather*}
  \frac{dr}{ar^3}=dt\\
  \int \frac{dr}{ar^3}=\int dt\\
  -\frac{1}{2ar^2}=t+C\\
  -\frac{1}{2ar^2}=t+C\\
  r(t)=\frac{1}{\sqrt{-2at+D}}
\end{gather*}
For the initial condition $r(0)=1$, we have $D=1$ and the resulting equation 
\[ r(t)=\frac{1}{\sqrt{-2at+1}} \]
experiences finite time blow up as $t\to 1$ given that $a>0$. Hence the origin is unstable.



\mathsolutioncheck{Correct}{2026-02-04 13:29}{The solution is mathematically correct throughout all parts. The equilibrium analysis, linearization, polar coordinate transformation, and stability analysis for all three cases ($a<0$, $a=0$, $a>0$) are all properly executed with valid reasoning and correct conclusions.}
\end{solution}

\end{document}
