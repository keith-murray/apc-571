\documentclass{article}
\usepackage{amsmath, amssymb}
\usepackage{mathhint}

\begin{document}

% Set the context for this problem - this information appears in the page header
% and helps the hint system understand what material you've covered
\mathhintcontext{
  book=Guckenheimer and Holmes,
  chapter=1,
  section=1.0-1.1,
  problem=5,
  lectures={lecture_1.tex, lecture_2.tex}
}

\begin{problem}
\noindent
(30 points) Use the candidate Liapunov function
\[ V(x,y)=\frac{1}{2}y^2+(1-\cos x) \]
to show that the origin $(x,y)=(0,0)$ is a locally asymptotically stable fixed point for the system
\begin{align*}
  \dot{x}&=y\\
  \dot{y}&=-\epsilon y^3-\sin x
\end{align*}
whenever $\epsilon>0$.

\end{problem}

\begin{notes}
Let's start by showing that our candidate Liapunov function has the desired properties. To show that $V(0,0)=0$, we write 
\[ V(0,0)=\frac{1}{2}(0)^2+(1-\cos 0)=0+(1-1)=0. \] 

To show that $V(x)>0$ for all $x\neq x_0$, let's consider a region $x\in[-\frac{\pi}{2},\frac{\pi}{2}]$. Our $y$ portion of $V$ will remain strictly positive when $y\neq 0$, and our $x$ portion ($1-\cos x$) will remain strictly positive when $x\in[-\frac{\pi}{2},\frac{\pi}{2}]$. Hence, $V(x)>0$ for all $x\neq x_0$.

To show that $\dot{V}(x)\leq 0$ for $x\neq x_0$, let's take the derivative. We write 
\begin{align*}
  \frac{d}{dt}V(x(t))&=\sum_{i=1}^{n}\frac{\partial V}{\partial x_i}\dot{x}_i\\
  &=(\sin x)\dot{x}+y\dot{y}\\
  &=(\sin x)y+y(-\epsilon y^3-\sin x)\\
  &=(\sin x)y -\epsilon y^4 - (\sin x)y\\
  &=-\epsilon y^4 
\end{align*}
and given that $\epsilon>0$, we can conclude that $\dot{V}=-\epsilon y^4\leq 0$. However, this only proves that our system is Lyapunov stable and not asymptotically stable. To show asymptotic stability, we can invoke La Salle's invariance principle, as we did in lecture 2.

More explicitly, $\dot{V(x,y)=0}$ when $y=0$. However, when $x\neq0$ and we consider the region $x\in[-\frac{\pi}{2},\frac{\pi}{2}]$, we have $\dot{y}=-sin x\neq 0$. Hence, our system dynamics are not invariant and do not map to another state where $\dot{V}=0$. Therefore, $V$ continues to decrease and is bounded below by $0$, implying that $V\to 0$ as $t\to \infty$. Therefore, $(x(t),y(t))\to 0$ and the origin is locally\footnote{Since we are only considering $x\in[-\frac{\pi}{2},\frac{\pi}{2}]$.} asymptotically stable.

\mathhint{Nudge}{2026-02-02 08:39}{Your analysis is quite thorough and you're on the right track with using LaSalle's invariance principle! However, take another careful look at your argument about the invariant set - you mentioned that when $y = 0$ and $x \neq 0$, we have $\dot{y} = -\sin x \neq 0$, but think about what this actually tells us about trajectories that start in the set where $\dot{V} = 0$.}

Following the hint, a more precise argument would be to consider the set 
\[ S=\left\{(x,0): x\in\left[-\frac{\pi}{2},0\right)\cup\left(0,\frac{\pi}{2}\right]\right\} \]
where it follows that $\dot{V}(x,y)=0$ for every $(x,y)\in S$. Yet, for trajectories that either start in or intersect the set, we have $\dot{y}=-\sin x\neq 0$, implying that part of the future trajector will leave the set. Hence the $S$ is not invariant under our system and we can invoke LaSalle's invariance principle to argue that the origin is a locally asymptotically stable fixed point.

\mathhint{Nudge}{2026-02-02 10:57}{Your refined analysis using LaSalle's invariance principle is much clearer now! You correctly identified that trajectories cannot remain in the set $S$ where $\dot{V} = 0$. Just double-check that your definition of the set $S$ includes all points where $\dot{V} = 0$ - you might want to think about what happens at the origin itself.}

\paragraph{Solution attempt}

Let's start by showing that our candidate Liapunov function has the desired properties. To show that $V(0,0)=0$, we write 
\[ V(0,0)=\frac{1}{2}(0)^2+(1-\cos 0)=0+(1-1)=0. \] 

To show that $V(x)>0$ for all $x\neq x_0$, let's consider a region where $x\in[-\frac{\pi}{2},\frac{\pi}{2}]$. Our $y$ portion of $V$ will remain strictly positive when $y\neq 0$, and our $x$ portion ($1-\cos x$) will remain strictly positive when $x\in[-\frac{\pi}{2},\frac{\pi}{2}]$. Hence, $V(x)>0$ for all $x\neq x_0$.

To show that $\dot{V}(x)\leq 0$ for $x\neq x_0$, let's take the derivative. We write 
\begin{align*}
  \frac{d}{dt}V(x(t))&=\sum_{i=1}^{n}\frac{\partial V}{\partial x_i}\dot{x}_i\\
  &=(\sin x)\dot{x}+y\dot{y}\\
  &=(\sin x)y+y(-\epsilon y^3-\sin x)\\
  &=(\sin x)y -\epsilon y^4 - (\sin x)y\\
  &=-\epsilon y^4 
\end{align*}
and given that $\epsilon>0$, we can conclude that $\dot{V}=-\epsilon y^4\leq 0$. However, this only proves that our system is Lyapunov stable and not asymptotically stable. To show asymptotic stability, we can invoke La Salle's invariance principle, as we did in lecture 2.

Consider the set 
\[ S=\left\{(x,0): x\in\left[-\frac{\pi}{2},\frac{\pi}{2}\right]\right\} \]
where it follows that $\dot{V}(x,y)=0$ for every $(x,y)\in S$. Yet, for trajectories that either start in or intersect the set at a point where $x\neq 0$, we have $\dot{y}=-\sin x\neq 0$, implying that part of the future trajector will leave the set. Hence, the only trajectory in $S$ that is invariant under our system is the trivial trajectory of $(x(0),y(0))=(0,0)$, and we can invoke LaSalle's invariance principle to argue that the origin is a locally asymptotically stable fixed point.

\mathsolutioncheck{Partial}{2026-02-02 11:04}{The approach using the Lyapunov function is correct and most calculations are accurate. However, there's an error in the domain restriction - the function $V(x,y) = \frac{1}{2}y^2 + (1-\cos x)$ is positive definite in a full neighborhood of the origin, not just $[-\pi/2, \pi/2]$, since $1-\cos x > 0$ for all $x \neq 0$ in any neighborhood of 0.}

Wait, how is $V(x,y) = \frac{1}{2}y^2 + (1-\cos x)$ is positive definite in a full neighborhood of the origin? What about $V(2\pi,0)$? Isn't $V(2\pi,0)=0$?

\mathhint{Nudge}{2026-02-02 12:31}{You're absolutely right to question that - $V(2\pi, 0) = 0$ indeed! The key insight is that for local asymptotic stability, we only need $V$ to be positive definite in some neighborhood of the origin, not globally.}

\end{notes}

\begin{solution}

Let's start by showing that our candidate Liapunov function has the desired properties. To show that $V(0,0)=0$, we write 
\[ V(0,0)=\frac{1}{2}(0)^2+(1-\cos 0)=0+(1-1)=0. \] 

To show that $V(x)>0$ for all $x\neq x_0$, let's consider a region
\[ U=\left\{ (x,y)\in\mathbb{R}^2 : x\in[-\frac{\pi}{2},\frac{\pi}{2}] \text{ and } y\in[-1,1] \right\} \]
in order to exclude points like $(2\pi,0)$ where $V(2\pi,0)=0$. Our $y$ portion of $V$ will remain strictly positive when $y\neq 0$, and our $x$ portion ($1-\cos x$) will remain strictly positive when $x\in[-\frac{\pi}{2},\frac{\pi}{2}]$. Hence, $V(x,y)>0$ for all $(x,y)\neq (0,0)$.

To show that $\dot{V}(x,y)\leq 0$ for $(x,y)\neq (0,0)$, let's take the derivative. We write 
\begin{align*}
  \frac{d}{dt}V(x(t), y(t))&=\sum_{i=1}^{n}\frac{\partial V}{\partial x_i}\dot{x}_i\\
  &=(\sin x)\dot{x}+y\dot{y}\\
  &=(\sin x)y+y(-\epsilon y^3-\sin x)\\
  &=(\sin x)y -\epsilon y^4 - (\sin x)y\\
  &=-\epsilon y^4 
\end{align*}
and given that $\epsilon>0$, we can conclude that $\dot{V}=-\epsilon y^4\leq 0$. However, this only proves that our system is Lyapunov stable and not asymptotically stable. To show asymptotic stability, we can invoke La Salle's invariance principle, as we did in lecture 2.

Consider the set 
\[ S=\left\{(x,0): x\in\left[-\frac{\pi}{2},\frac{\pi}{2}\right]\right\} \]
where it follows that $\dot{V}(x,y)=0$ for every $(x,y)\in S$. Yet, for trajectories that either start in or intersect the set at a point where $x\neq 0$, we have $\dot{y}=-\sin x\neq 0$, implying that part of the future trajector will leave the set. Hence, the only trajectory in $S$ that is invariant under our system is the trivial trajectory of $(x(0),y(0))=(0,0)$, and we can invoke LaSalle's invariance principle to argue that the origin is a locally asymptotically stable fixed point.

\mathsolutioncheck{Partial}{2026-02-02 12:39}{The approach using the Lyapunov function is correct and most calculations are accurate. However, there's an error in the analysis of the invariant set $S$ - when $(x,y) \in S$ with $x \neq 0$, we have $\dot{y} = -\sin x \neq 0$, which means trajectories leave $S$ immediately, so the largest invariant subset of $S$ is indeed just the origin.}

\end{solution}

\end{document}
