\documentclass{article}
\usepackage{amsmath, amssymb}

\begin{document}

\section{Plan for Class}
\begin{enumerate}
    \item Definitions in \S 1.6
    \item Planar systems
\end{enumerate}

\section{Definitions in \S 1.6}

\subsection{Non-wandering point}
\textit{Definition:} $p$ is a non-wondering point for a flow $\varphi_t$ / map $F$ if $\forall$ neighborhoods $p\in U$, all $T>0$
\begin{itemize}
    \item $\varphi_t(U)\bigcap U$ is non-empty for some $t>T$.
    \item $F^t(U)\bigcap U$ is non-empty for some $t>T$.
\end{itemize}

This is just a formalization of recurrence. For example, fixed points and periodic orbits satisfy this definition. But then, why have this definition if we already have `fixed points' and `periodic orbits'. There are more exotic examples! Poincare recurrence\footnote{Did we define this?} is an example of a non-wondering point.

\subsubsection{Example}
Let's say we have the following system on a torus
\begin{align*}
    \dot{\theta}&=1\\
    \dot{\varphi}&=\sqrt{2}
\end{align*}
In this system, since $\dot{\varphi}$ is irrational, then every point is non-wondering. Well actually $\dot{\varphi}$ could be rational and every point is still non-wondering, but just not densely.

\subsection{Limit point}
\textit{Definition:} $p$ is an $\Big|^\omega_\alpha$ limit point of $x$ if $\exists$ points 
\begin{itemize}
    \item $\omega$ limit point is $\varphi_{t_j}\to p$ as $t_j\to \infty$
    \item $\alpha$ limit point is $\varphi_{t_j}\to p$ as $t_j\to - \infty$
\end{itemize}
where $t_j$ is a subsequence of times. 

Set of all $\Big|^\omega_\alpha$ limit points of $x$ is the $\Big|^\omega_\alpha$ limit set of $x$.

\subsection{Dense orbit}
\textit{Definition:} An orbit $\varphi_{t}(q)$ is dense in a set $S\subset \mathbb{R}^n$ if $\forall p\in S$ and $\forall \epsilon >0$ there exists $t$ such that $\big|\big| \varphi_{t}(q)-p \big|\big| < \epsilon$. 

In other words $\varphi_{t}(q)$ comes arbitrarily close to every point in $S$. Look at the $\dot{\varphi}=\sqrt{2}$ for an example of dense orbits and $\dot{\varphi}=2$ for non dense orbits.

\subsection{Attractor}
\textit{Provisional Definition:} An attractor is an attracting set with a dense orbit.

In a previous lecture, we had an attracting set but no an attractor becuase there was no dense orbit.


\section{Back to ODEs/flows (not maps)}
If we are in 1 dimensions, we can't have oscillations or periodic orbits. Fixed points are the only non-wandering points. BORING! Of course, the exception is the circle where we can have periodic orbits.

In 2 dimensions, more things are possible. We can have
\begin{itemize}
    \item periodic orbits 
    \item homoclinic connections
    \item heteroclinic connections 
\end{itemize}
Although, heteroclinic connections are possible in 1 dimensions, but those are stable under perturbations. Heteroclinic connections are ``fragile'' in 2 dimensions.

\subsection{Poincare-Bendixion theorem}
\textit{Theorem:} A nonempty compact $\omega$ or $\alpha$ limit set of a planar flow, which contains no fixed points, is a closed orbit.

In other words, suppose $f$ is $C^1$ and has a trajectory $\Gamma$ with $\varphi_t(\Gamma)$ contained in a compact set. Then if $\omega(\Gamma)$ contains no fixed point, $\omega(\Gamma)$ is a periodic orbit.

\subsubsection{Examples}
Suppose there are a finite number of fixed points in a compact set $S$. Then $\omega(\Gamma)$ is one of the following:
\begin{enumerate}
    \item[(a)] an equilibrium point 
    \item[(b)] periodic orbit 
    \item[(c)] collection of fixed points and orbits connecting them 
\end{enumerate}

\subsubsection{Another example}
Prove that 
\begin{align*}
    \dot{x}&=x-y-x^3\\
    \dot{y}&=x+y-y^3\\
\end{align*}
has a periodic orbit in the region $1<r<\sqrt{2}$ where $r^2=x^2+y^2$.

There is only one fixed point at $(x,y)=(0,0)$. If we show that $\dot{r}>0$ at $r=1$ and that $\dot{r}<0$ at $r=\sqrt{2}$, AND that there are no fixed points in $1<r<\sqrt{2}$, then Poincare-Bendixion theorem implies that there is a periodic orbit.

Nice.

\subsection{Bendixion's criterion}
This criterion will let us rule out periodic orbits. Let's say 
\begin{align*}
    \dot{x}&=f(x,y)\\
    \dot{y}&=g(x,y)
\end{align*}
\textit{Theorem:} Suppose $D\subset \mathbb{R}^2$ is simply connected (no holes). Then the system above can have periodic solution only if $\nabla\cdot(f,g)=\partial_x f + \partial_y g$ changes sign or is identically zero in $D$.

\subsubsection{Proof}
Suppose $\Gamma$ is a periodic orbit in $D$. Let $S$ denote the interior of $\Gamma$. Consider Green's theorem 
\begin{align*}
    \iint_S \nabla\cdot (f,g)d\sigma &=\int_\Gamma \left(f\, dy - g\, dx\right)\\
    &=\int_\Gamma \left(f\, \frac{dy}{dt} - g\, \frac{dx}{dt}\right)\,dt\\
    &=\int_\Gamma \left(fg-gf\right)\,dt=0
\end{align*}
Hence the integrand $\iint_S \nabla\cdot (f,g)d\sigma$ must change sign in $S$ or be equivalent to 0 in $S$.

\paragraph{Note}
Simply connected is important! For example 
\begin{align*}
    f&=x-y-x^3\\
    g&=x+y-y^3
\end{align*}
We have 
\begin{align*}
    f_x&=1-3x^2\\
    g_y&=1-3y^2
\end{align*}
then 
\[ f_x+g_y=2-3(x^2+y^2)=2-3r^2<0 \]
in our annulus!

\subsection{Index Theory}
If you are following along, this is page 51-53.

Suppose 
\[ \dot{x}=f(x)\qquad x\in\mathbb{R}^2 \]
Index of a close curve $C$ equals the number of turns vector field $f$ makes as $C$ is traversed conunter-clockwise.

This is made clear by the expression 
\[
k(C)=\frac{1}{2\pi}\int_C d\, \arctan\left(\frac{dy}{dx}\right)=\frac{1}{2\pi}\int_C \frac{f\,dy-g\,dx}{f^2+g^2}
\]

\subsubsection{Examples}
\begin{itemize}
    \item A sink has an index of $+1$
    \item A source has an index of $+1$
    \item An outward spiral has an index of $+1$
    \item An inward spiral has an index of $+1$
\end{itemize}
In general, a periodic orbit has an index of $+1$. Well, what doesn't have an index of $+1$?

A saddle has an index of $-1$. Maybe I should provide the drawing.

A region containing no fixed points has an index of $0$.

The idea is that the index of $C$ is the sum of the indices all the fixed points inside $C$.

\subsubsection{Corollary}
Inside any closed orbit (periodic orbit as(is?) limit cycle), there exists at least one fixed point (in fact $\sum$ind(fd)=1), so if there's only one, it must be a source or sink.\footnote{This is terribly worded.}

\paragraph{Note:}
Degenerate fixed points may have indices not equal to $\pm 1$. For example, we have the system 
\begin{align*}
    \dot{x}&=x^2-y^2\\
    \dot{y}&=2xy
\end{align*}
we can define $z=x+iy$ where 
\begin{align*}
    \dot{z}=z^2&=(x+iy)(x+iy)\\
    &=(x^2+y^2)+2ixy 
\end{align*}
huh that's weird, but works! The index of the system is $+2$. In general, if $\dot{z}=z^k$, then the system has index $k$.

\subsection{Hamiltonian systems}
Let's look at Hamiltonian systems not restricted to $\mathbb{R}^2$. In general, a Hamiltonian is $H:\mathbb{R}^2\longrightarrow \mathbb{R}$, which is typically ``energy'', and satisfies 
\begin{align*}
    \dot{x}&=\frac{\partial H}{\partial y}\\
    \dot{y}&=-\frac{\partial H}{\partial x}
\end{align*}
We can extend this to multiple dimensions as 
\begin{align*}
    \dot{x}_i&=\frac{\partial H}{\partial y_i}\\
    \dot{y}_i&=-\frac{\partial H}{\partial x_i}
\end{align*}
where $i=1,\ldots,n$ and $x$ is interpreted as position and $y$ is interpreted as momentum.

We've already shown in class how $H$ is conserved along trajectories because $\frac{dH}{dt}=0$. Hence solution curves lie on level sets of $H$.

\subsubsection{Example: simple pendulum}
Note that $p=$momentum and can be written as $p=ml^2\dot(\theta)$. Hence we have 
\begin{align*}
    H(\theta,p)&=\text{ total energy}\\
    &=\frac{1}{2}ml^2\dot{\theta}^2+mgl(1-\cos\theta)\\
    &=\frac{p^2}{2ml^2}+mgl(1-\cos\theta)
\end{align*}
Hence we have 
\begin{align*}
    \dot{\theta}&=\frac{\partial H}{\partial p}=\frac{p^2}{ml^2}\\
    \dot{p}&=-\frac{\partial H}{\partial \theta}=-mgl\sin\theta
\end{align*}




\end{document}